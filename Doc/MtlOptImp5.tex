\documentclass[9pt,twocolumn]{extarticle}

\usepackage[hmargin=0.5in,tmargin=0.5in]{geometry}
\usepackage{amsmath,amssymb}
\usepackage{times}
\usepackage{graphicx}
\usepackage{subfigure}

\usepackage{cleveref}
\usepackage{color}
\newcommand{\TODO}[1]{\textcolor{red}{#1}}

\newcommand{\FPP}[2]{\frac{\partial #1}{\partial #2}}
\newcommand{\argmin}{\operatornamewithlimits{arg\ min}}
\author{Siwang Li}

\title{Experiments results of Material optimization}

%% document begin here
\begin{document}
\maketitle

\setlength{\parskip}{0.5ex}

\section{Experiments}
We use five keyframes as shown in figure \ref{rlst}(a) to make the experiments
with and without material optimization. With material optimization, We hope to
generate one mode, which behaviors similar to these keyframes, e.g swing left
half part, with the right half part fixed. We use 6 and 10 modes to produce the
results using material optimization, and 10 modes without material optimization,
the results are shown in figure \ref{rlst}(b) and the video. The curves for $z$
are shown figure \ref{curves-z}, and the optimization process are shown in
figure \ref{opt-6} and \ref{opt-10}.

\begin{figure}
  \centering
  \subfigure[keyframes] { \label{fig:b}
    \includegraphics[width=0.4\textwidth]{./figures/beam-fix-end-center-keyf.png}
  }
  \subfigure[results.] { \label{fig:b}
    \includegraphics[width=0.4\textwidth]{./figures/beam-fix-end-center.png}
  }
  \caption{(a)Keyframes at frame 0,10,20,24,44, with the right half parts are
    fixed. (b)Frame of resulting animation at frame 44.}
  \label{rlst}
\end{figure}

\section{Analysis}
\begin{itemize}
\item If the modes is enough, we can obtain the desired modes using material
optimization(see figure \ref{rlst}(b)).
\item By only optimizing the $K$ in subspace, it is hard to generate a mode with
  exactly one side fixed, and it will introduce some motions with high
  frequency, especially when there are many modes to be optimized, as shown in
  figure \ref{curves-z} and the video. 
\end{itemize}
We hope to improve the results by introducing the optimization of the modes $W$
and rest shape $r$.

\begin{figure}
  \centering
  \subfigure[no material optimization,10 modes.] { \label{fig:b}
  \includegraphics[width=0.4\textwidth]{./figures/mtlopt_cen_keyf_swing_one_end_10_modes_Opt_Z_curveZ.png}}
  \subfigure[material optimization,6 modes.] { \label{fig:b}
  \includegraphics[width=0.4\textwidth]{./figures/mtlopt_cen_keyf_swing_one_end_06_modes_Opt_Z_AtA_curveZ.png}}
  \subfigure[material optimization,10 modes.] { \label{fig:b}
  \includegraphics[width=0.4\textwidth]{./figures/mtlopt_cen_keyf_swing_one_end_10_modes_Opt_Z_AtA_curveZ.png}}
  \caption{Curves of $z$ with and without material optimization.}
  \label{curves-z}
\end{figure}

\begin{figure}
  \centering
  \subfigure[Inner iteration.] { \label{fig:b}
  \includegraphics[width=0.48\textwidth]{./figures/mtlopt_cen_keyf_swing_one_end_06_modesOpt_Z_AtA-inner_py.png}}
  \subfigure[Outer iteration.] { \label{fig:b}
  \includegraphics[width=0.48\textwidth]{./figures/mtlopt_cen_keyf_swing_one_end_06_modesOpt_Z_AtA-outer_py.png}}
  \caption{Energy-iteration curves for the material optimization with 6 modes.}
  \label{opt-6}
\end{figure}


\begin{figure}
  \centering
  \subfigure[Inner iteration.] { \label{fig:b}
  \includegraphics[width=0.48\textwidth]{./figures/mtlopt_cen_keyf_swing_one_end_10_modesOpt_Z_AtA-inner_py.png}}
  \subfigure[Outer iteration.] { \label{fig:b}
  \includegraphics[width=0.48\textwidth]{./figures/mtlopt_cen_keyf_swing_one_end_10_modesOpt_Z_AtA-outer_py.png}}
  \caption{Energy-iteration curves for the material optimization with 10 modes.}
  \label{opt-10}
\end{figure}

\end{document}