\documentclass[9pt,twocolumn]{extarticle}

\usepackage[hmargin=0.5in,tmargin=0.5in]{geometry}
\usepackage{amsmath,amssymb}
\usepackage{times}
\usepackage{graphicx}
\usepackage{subfigure}

\usepackage{cleveref}
\usepackage{color}
\newcommand{\TODO}[1]{\textcolor{red}{#1}}

\newcommand{\FPP}[2]{\frac{\partial #1}{\partial #2}}
\newcommand{\argmin}{\operatornamewithlimits{arg\ min}}
\author{Siwang Li}

\title{Report for Space-time Material Optimization}

%% document begin here
\begin{document}
\maketitle

\setlength{\parskip}{0.5ex}
\begin{abstract}
  Given a motion equation, the corresponding initial elastic materials, and some
  partial constraints provided by the user, the approach proposed here can
  generate an animation sequence satisfied these partial constraints, with
  minimal control forces and consistent effects. The novelties of this work
  include, an efficient method for the space-time elastic materials optimization
  in the subspace and an efficient Reduced RS model reduction method to support
  large deformation and promise tightly positional constraints. What's more, by
  providing a novel method for basis optimization(selection/combination), this
  approach can also automatically find a compact subspace to represent the
  resulting animation sequence.
\end{abstract}

\section{Introduction}
Traditional method to create an animation is to interpolate a set of
key frames.  Two problems here:
\begin{itemize}
\item User needs to provide a set of key frames which describes the whole shape
  of the model at specific times.  We argue that incomplete key frame
  (positional constraints) could improve the efficiency of animation authoring
  and quality of the final results.
\item The interpolation method often leads to over-smoothed result because of
  lacking proper dynamics effects.  Increasing the number of key frames can
  alleviate the problem, however, requires more tedious human work. Thus, we
  propose to integrate the equation of motion into the animation authoring
  procedure.
\end{itemize}

Recently, some methods have been proposed for physically based sequence editing,
which could be applied to create the animation as well. But, in these methods,
only moderate change is allowed, otherwise, there would be great distortion
making the result unnatural, or can not provide tightly positional constraints
which is required when the artist intend to design a portion of a shape,
i.e. partial constraints. More importantly, they require the input sequence
coming with a proper equation of motion, e.g. the stiffness, mass etc.
Unsuitable material setting leads large control force.  Be more specific, there
will be the following typical problems:
\begin{itemize}
\item The dynamics around the editing frame is not consistent with the
  other frames far away from the editing frame.
\item Repeatedly editing is required for desired periodically
  behavior.
\end{itemize}

However, a suitable material setting is not easy to get in general, and such a
requirement imposes additional difficulty than the traditional key frame based
method. In this paper, we propose a method to progressively create an animation
by space-time constraints, which is driven by an equation of motion with
material to be optimized. 

In the following, we will first review some background techniques in section
\ref{sec:background}, then introduce our Reduced-RS method in section
\ref{sec:reduced-rs-method} , and our material optimization scheme in section
\ref{sec:mater-optim}, finally the experiments' results in section
\ref{sec:experiments}, and conclusion in section \ref{sec:conclusion}.

\section{Background}\label{sec:background}
Firstly, our framework is based on space-time optimization on a set of decoupled
linear motion equations, which is closely related to modal analysis. Secondly,
we adopt the RS-coordinates to avoid the artifacts under large deformation. In
this section, we briefly review these techniques.

\subsection{Modal analysis}
The linear motion equation of an object is
\begin{equation} \label{linear_eq}
  \tilde{M}\ddot{u} + (\alpha_mI+\alpha_k\tilde{K})\dot{u} + \tilde{K}u = f_{ext}(t)
\end{equation}
where $u\in R^{3n}$ represents the displacements of all the $n$ nodes of the
object, $\tilde{M},\tilde{K} \in R^{3n\times 3n}$ are the mass and stiffness
matrix respectively, $\alpha_m,\alpha_k$ are the damping coefficients, and
$f_{ext}(t)$ is the external force at time $t$.

By using modal analysis method, motion equation (\ref{linear_eq}) can be reduced
and decoupled. To achieve this, we solve a general eigenvalue problem
\begin{equation} \label{general_eig}
  \tilde{K}\phi = \lambda \tilde{M}\phi
\end{equation}
If we preserve the first $r_w$ eigenvectors and eigenvalues to obtain
$W=(\phi_1,\cdots,\phi_{r_w})$, and $\Lambda=diag(\lambda_1,\lambda_{r_w})$,
then replace $u=Wz$ in equation (\ref{linear_eq}) and pre-multiply it with
$W^T$, finally we obtain the decoupled reduced motion equations
\begin{equation} \label{ma_eq}
  \ddot{z} + (\alpha_mI+\alpha_k\Lambda)\dot{z} + \Lambda{z} = w(t)
\end{equation}
Here, $w(t)=W^Tf_{ext}(t)$ represents the forces in the subspace.

\subsection{RS method}
Given a displacement $z\in R^{r_w}$ in the frequency subspace, there are many
method to construct the displacements $u(z)\in R^{3n}$ in full space, such as
linear mapping (e.g $u = Wz$), modal warping, or RS coordinates. Experiments
show that, when the deformation is large, only the RS coordinates method can
produce plausible results. In the following, we briefly review the RS method,
and we refer it as full-RS here. Using full-RS, we compute $u(z)$ by solving
\begin{equation}\label{full_rs_min}
  \min_{u} E_{rs}(u)=\frac{1}{2}\sum_{e\in T_{all}}\|(Gu)_e + I - \exp({y_e^w})(I+y_e^s)\|_2^2|V_e|
\end{equation}
where $T_{all}$ denotes all of the tetrahedrons, $G$ is the discrete gradient
operator with respect to the rest shape, $|V_e|$ is the volume of tetrahedron
$e$, $y_e=(y_e^w,y_e^s)$ is the RS coordinates of tetrahedron $e$. We denote the
RS-coordinates $y=(y_1,\cdots,y_{|T_{all}|})$, and compute it using
$y=\hat{W}z=(QW)z$, where $Q$ is a sparse matrix which maps the basis $W$ to the
RS-basis $\hat{W}$. Then the solution of (\ref{full_rs_min}) can be found by
solving
\begin{equation} \label{rs}
  Au = b(y(z)).
\end{equation}
where $A=G^TVG$ is a constant sparse matrix($V$ is a diagonal matrix consisted
of $V_e$), and $b(y(z))=-\frac{\partial{E_{rs}(u)}}{\partial{u}}|_{u=0}$ is a
nonlinear function.

\section{Reduced RS method}\label{sec:reduced-rs-method}
As in our spacetime optimization with positional constraints(introduced in
section \ref{sec:mater-optim}), we need to compute $u_j(z)$ and
$\frac{\partial{u_j(z)}}{\partial{z}}$ for each constrained node $j$
frequently. Using the full RS method introduced above, these computation would
be expensive. In this section, we presents a reduced RS method to improve the
performance.

Firstly, we assume $u$ can be well approximated by using some basis matrix $B\in
R^{3n\times r_b}$, then we replace $u = Bq$ into (\ref{rs}), and multiply
$B^T$in both sides to obtain
\begin{equation} \label{reduced_rs1} B^T(ABq) = B^Tb(y(z))
\end{equation} Let $P = (B^TAB)^{-1}B^T$, then we can compute the reduced
coordinates $q$ using
\begin{equation} \label{reduced_rs2} 
  q = Pb(y) = \sum_{e\in{T_{all}}}P_e{b_e(y_e(z))}
\end{equation} where $P_e\in R^{r_b\times 9}$ is some constant matrix, $b_e(\cdot)\in
R^{9}$ is some nonlinear function of $y_e$, and $y_e\in R^9$ is
the RS coordinates of tetrahedron $e$. 

Then, we approximate $q$ using the cubature scheme, and obtain
\begin{equation} \label{reduced_rs3} 
  q \approx \sum_{e\in{T_{cub.}}}w_eP_e{b_e(y_e(z))}
\end{equation} 
where $w_e$ is the cubature weights, and $T_{cub.}$ is the set of the selected
cubature tetrahedrons. Finally, we can compute the displacements $u$ using
\begin{equation} \label{reduced_rs4}
  u(z) = B \sum_{e\in{T_{cub.}}}w_eP_e{b_e(y_e(z))}
\end{equation} 
which is much faster than solving the large linear system (\ref{rs}).

\subsection{Performance analysis} 
To compute $u_j(z)$ and $\frac{\partial{u_j(z)}}{\partial{z}}$ for some node $j$
with the Reduced-RS method, we only need to compute $y_e(z)$ and $b_e(y_e)$ for
the sampled tetrahedrons (usually less than 100), and then make some small
matrix-vector production to obtain the results. While using full-RS method, to
compute $u_j(z)$, one needs to compute $y$ and $b$ for all the tetrahedrons, and
finally solve a large sparse linear system to compute $u$ for all the nodes, and
select $u_j(z)$ from them. And use full-RS method to compute
$\frac{\partial{u_j(z)}}{\partial{z}}$ would be much expensive than compute
$u_j(z)$. Thus, our reduced method should be much efficient than full-RS method,
especially when $u_j(z)$ and $\frac{\partial{u_j(z)}}{\partial{z}}$ need to be
computed frequently, such as the sparse positional constraints in our spacetime
optimization introduced bellow.

\section{Material optimization}\label{sec:mater-optim}
In the following, we will introduce our spacetime material optimization scheme.
Firstly, we choose the basis $B,W$ and cubature samples $T_{cub}$ to be large
enough to represent the desired deformations. We can use modal derivative method
to obtain $B$, and use modal analysis method to produce $W$. Then we further
assume that, the resulting animation can usually be represented by some much
smaller basis $W'=WS$, where $S\in R^{r_w\times r_s}$ is a dense matrix with
$r_s$ is much smaller than $r_w$. For example, we can choose $r_w=50$, while
$r_s=10$. There is no necessary to require that $W'$ to be mass-orthogonal, as
the mass matrix is unknown.

Now, in the subspace with basis in $W'$, we assume that the motion of the object
to be edited can be described by this decoupled motion equation
\begin{equation} \label{ma_eq_1}
  \ddot{z} + (\alpha_mI+\alpha_k\Lambda)\dot{z} + \Lambda{z} = w(t)
\end{equation}
with $w(t)$ is the control forces.

Given some user defined constraints, we define our spacetime material
optimization problem as
\begin{eqnarray}\label{mtl-opt}
  &&\argmin_{z,\Lambda,S} E_w(z,\Lambda)+\gamma E_c(S,z)+\mu E_s(S)\\
  &&\mbox {subject to } \Lambda_i \ge 0
\end{eqnarray}

Here, $\gamma$ and $\mu$ are the penalties, and $E_w$ is the residual for the
control forces $w(t)$, $E_c$ represents the residual of the constraints, while
$E_s$ is a regularization term for $S$. We will define each of these energies in
the following. 

\paragraph{Residual of control forces.} To define $E_w$, we discretize the
motion equation (\ref{ma_eq_1}) in time, and define it using spacetime
constraints
\begin{equation} \label{ew}
  E_w(z,\Lambda) = \frac{1}{2}\sum_{i=1}^{T-2}\|\hat{z}_i+(\alpha_mI+\alpha_k\Lambda)(z_{i+1}-z_{i-1})+\Lambda z_i\|_2^2
\end{equation}
where $\hat{z}_i=z_{i+1}-2z_{i}+z_{i-1}$, and $T$ is the total number of frames.

\paragraph{Residual for constraints. } The user provided partial constraints can
be defined as $\{C_iu_i(S,z)=u^c_i\}_{i\in \mathbb{C}}$, where $\mathbb{C}$
denotes the constrained frames, and $C_i$ is the constraint matrix for frame
$i$, while $u^c_i$ represents the target positions. Then we define $E_c$ as an
quadratic energy function to penalize it as
\begin{equation} \label{ec}
  E_c(z,S) = \frac{1}{2}\sum_{i\in \mathbb{C}}\|C_iu_i(S,z)-u^c_i\|_2^2
\end{equation}
Here, we compute $u(S,z_i)$ using our Reduced-RS method,
\begin{equation} \label{redrs_s}
  u(S,z_i) = B\sum_{e\in T_{cub}}w_eP_eb_e(\hat{W}_eSz_i)
\end{equation}
where $\hat{W}_e\in R^{12\times r_w}$ is the sub-matrix of the RS-basis
$\hat{W}$ with respect to tetrahedron $e$. 

\paragraph{Regularization of $S$. } As the modes in $W$ with smaller frequencies
are usually much smooth than the modes with high frequencies, we adopt a
regularization term to penalize each rows of $S$ using the corresponding initial
frequency $\lambda_i^0$, i.e.
\begin{equation} \label{reg_s}
  E_s(S) = \sum_{i=0}^{r_w-1}\sum_{j=0}^{r_s-1}\lambda_i S^2_{i j}
\end{equation}
Here, $\lambda_i^0$ is eigenvalues obtained from the initial setting materials
by solving the general eigenvalue problem (\ref{general_eig}).

\begin{center}
  \begin{table*}[ht]
    {\small
      \hfill{}
      \begin{tabular}{ | l | c | c | c | c | c | c | c| c | c | c | c | c |}\hline
	    model & animation& $T$ &$T_{all}$ & $T_{cub}$& $r_b$ & $r_w$& $r_s$ &mtl opt& z opt \\ \hline
        beam & swing two ends. &200&  1072 &100 &80 &   10& 5& $1.1\times 10^6$& $1.0\times 10^7$ \\ \hline
        beam & swing one end.  &200&  1072 &97  &80 &  10&  5& $6.1\times 10^5$& $3.3\times 10^7$ \\ \hline        
        mushroom& rotate.      &400&  2388 &98  &80 &   6&  3& $8.2\times 10^4$& $7.1\times 10^6$  \\ \hline
      \end{tabular}
      \hfill{}
}
\caption{ Statistics for the examperiments.  From left to right: the name of the
  model(model), the desired animation(animation), total frames($T$), number of
  tetrahedrons($T_{all}$), number of cubature samples($T_{cub}$), columns of
  $B$($r_b$), $W$($r_w$) and $S$($r_s$), the optimal function value
  with(mtl opt) and without(z opt) material optimization.}
\label{table_1}
\end{table*}
\end{center}

\section{Experiments}\label{sec:experiments}
\paragraph{Setup. }In this section, we use several simple examples to
demonstrate our approach. Given the initial elastic materials, we compute the
full stiffness matrix $\tilde{K}$ and mass matrix $\tilde{M}$, then solve a
general eigenvalue problem (\ref{general_eig}) to obtain $W$ with the first
$r_w$ eigenvectors, and $\Lambda_0$ with the first $r_s$ eigenvalues. Then we
generate $B$ using modal derivative method. 
% To produce the cubature samples. 
When we solve for (\ref{mtl-opt}), we use $z=0,\Lambda=\Lambda_0$ and
$S=(I_{r_s\times r_s},O)^T$ as the initial value.

\paragraph{Implementation. } Currently, as we implement our spacetime material
optimization method using an auto-differential library(CASADI combined with
IPOPT), we can only produce some simple examples with no more than ten modes in
$W$. A much efficient version will be implemented soon, and more modes will be
supported.

\subsection{Comparison of the warping approaches}
W compare the results of using different methods to construct the displacement
$u$ with the same reduced displacement $z$. The results are shown in figure
\ref{rs_compare}. It can be seen that, when the deformation is large, the
Reduced-RS method and full-RS method can produce plausible results, while the
linear mapping and the modal warping method will produce noticeable
artifacts. In this experiment, we use $r_b=80,r_w=80$ and adopt $89$ cubature
samples.

\begin{figure}
  \centering \subfigure[rest shape(white) and result of linear mapping(purple)]
  { \label{fig:a}
    \includegraphics[width=0.225\textwidth]{./figures/redrs_ma.png}}
  \subfigure[modal warping ] { \label{fig:b}
    \includegraphics[width=0.225\textwidth]{./figures/redrs_mw.png}}
  \subfigure[full-RS ] { \label{fig:b}
    \includegraphics[width=0.225\textwidth]{./figures/redrs_fullrs.png}}
  \subfigure[Reduced-RS ] { \label{fig:b}
    \includegraphics[width=0.225\textwidth]{./figures/redrs_redrs.png}}
  \caption{Reconstruct $u$ from the same $z$ using different method. The white
    shape in (a) is the rest shape.}
  \label{rs_compare}
\end{figure}

\subsection{Material opt. with partial constraints}
In the following examples, we produce the resulting animation using two
approaches: using material optimization, and without material optimization, i.e
with only $z$ to be optimized. For the case without material optimization, we
solve the following optimization problem,
\begin{equation} \label{opt_z}
  \argmin_{z} E_w(z,\Lambda)+\gamma E_c(S,z)+\mu E_s(S)
\end{equation}
where we fix $\Lambda=\Lambda_0$, set $S=(I_{r_s\times r_s},O)^T$ to select the
first $r_s$ modes in $W$, and choose the same $\gamma$ and $\mu$ as for material
optimization.

The input for the first example is a static beam with the center fixed, and we
set $z_0=z_{1}=0$. There are four positional constraints applied at frame $104$
and frame $124$, which are designed to make the two ends of the beam swing up
and down symmetrically. As shown in table \ref{table_1}, by using materials
optimization, the optimal value for the objective function is much smaller,
which indicates that the control forces are much smaller, and the constraints
are satisfied much better. And as shown in the video and figure
\ref{result_z}(a), with materials optimization, the resulting animation can
produce interesting consistent effects. Furthermore, as shown in
figure\ref{result_z}(a), our approach automatically choose two modes (i.e. mode
0 and 1) to represent the resulting animation, while at least 4 modes are needed
when there are no materials optimization (figure\ref{result_z}(b)). This
indicates that, our method can automatically find a compact subspace to describe
the resulting motion.

In the second example, we modified a static mushroom to rotate using five
partial constraints at frame 0,50,100,150 and 200. The root of the mushroom is
fixed. As shown in the video and figure \ref{result_z}(e)(f), because the
initial material for the mushroom is soft, without material optimization it can
not repeat the rotation motion when there are no constraints(from frame 201 to
frame 399). And our approach finds the optimal material to produce this rotation
motion, avoid repeated editing.

\subsection{Material opt. with keyframes}
Sometimes, the artist hope to design or modify the animation using keyframes. If
the nodes of the object is too many, it would inefficient to constrained all the
nodes. One can choose to constraint only the nodes of the tetrahedrons that
contains the vertices of the embed mesh. Another approach is to choose the nodes
of the tetrahedrons with large cubature weights as partial constraints. In this
experiment, we use the second method.
% Furthermore, if the keyframes
% are given as surface mesh, or designed using some geometric-based approach, we
% need to firstly approximate it with a tetrahedral mesh.  using the approach
% described in \cite{sub_sp_opt}.

We use a beam as input to make the experiment. Several nodes at one end of the
beam is fixed. We set $z_0=z_{20}=0$, and use two tetrahedral mesh shown in
figure \ref{key_beam} as additional keyframes to generate the resulting
animation. The displacements of half part of the keyframes are zero, and we
intend to generate an animation with only one half of the beam is swing, while
another half part is almost static. We approximate both keyframes using 51 nodes
of the tetrahedrons with large cubature weights. Then we use these selected
points as partial constraints to produce the results with and without material
optimization. As shown in the video, the results produced by using material
optimization can produce the desired effects, i.e swing only one end. While
without material optimization, the resulting animation can not produce such
consistent effects, and it is even can not satisfy the constraints. 

\subsection{Material opt. using dense $K$}
In our earlier research stage, we propose to optimize for a dense SPD matrix
$K=A^TA$ instead of a diagonal matrix $\Lambda$ for equation (\ref{ew}), where
$A\in R^{r_s\times r_s}$. By using this approach, we need to solve
\begin{equation} \label{mtl-opt_dense_seq}
  \argmin_{z,A,S} E_w(z,K(A))+\gamma E_c(S,z)+\mu E_s(S)
\end{equation}
According to our experiments, such a scheme can produce much smaller values of
the objective function, i.e. smaller control forces with constraints well
satisfied. However, with such a scheme, the computational cost is much larger,
and more importantly, it may produce large distortion in the resulting
animation, especially when the constraints is hard to satisfied. Following is
the reason.
\begin{figure}
  \centering
  \subfigure[keyframe at frame 10] { \label{fig:a}
    \includegraphics[width=0.225\textwidth]{./figures/beam-swing-key1.png}}
  \subfigure[keyframe at frame 30] { \label{fig:b}
    \includegraphics[width=0.225\textwidth]{./figures/beam-swing-key3.png}}
  \caption{Keyframes for material optimization. Red points are the partial
    constraints used to approximate the keyframes.}
  \label{key_beam}
\end{figure}

As we require $K$ be a SPD matrix, we can decompose $K=U\Lambda' U^T$, with
$U$ is orthogonal, and $\Lambda'$ is a diagonal matrix, with $\Lambda'_i\ge
0$. If we replace $z=Uz'$, then problem (\ref{mtl-opt_dense_seq}) is equal to
\begin{equation} \label{mtl-opt_dense_2}
  \argmin_{z',\Lambda',S,U} E_w(z',\Lambda')+\gamma E_c(SU,z')+\mu E_s(S)
\end{equation}
with additional constraints that $U$ is orthogonal, and
$\Lambda'_i\ge0$. Compare to (\ref{mtl-opt}), it have not applied regularization
terms on $U$ using the initial eigenvalues. Thus it would more easily to
introduce the modes with large frequency into the resulting animation, and if
the amplitude of such modes is too large, the resulting animation would produce
non-smooth motion, as shown in the video.

% \subsection{Performance}

\section{Conclusion}\label{sec:conclusion}
Firstly, experiments shows that, with the Reduced-RS approach introduced here,
we can reconstruct shape in full space from the decoupled subspace efficiently,
and the deformation is plausible. Such an approach has helped us to efficiently
solve the spacetime optimal control problem with partial constraints. Secondly,
by using material optimization, the control forces introduced in the spacetime
optimal control can be significantly reduced, the constraints can be satisfied
much better, and we can observe some interesting consistent effects. We propose
to optimize the basis by introduce a basis selection matrix, and such a scheme
not only reduce the control forces, but also find some compact subspace to
represent the resulting animation.

\paragraph{Remaining problems.} The first problem is the efficient of the
algorithm, including both the numerical approach and the implementation. We need
to find some efficient numerical method to solve our objective function
(\ref{mtl-opt}) efficiently. We hope to achieve interactive
performance. Secondly, we need to find out more interesting applications.

\begin{figure}
  \centering
  \subfigure[mtl opt] { \label{fig:a}
    \includegraphics[width=0.225\textwidth]{./figures/mtloptW\string_510mode\string_dual\string_p2-iniOpt\string_Z\string_Lambda\string_updateW-mtllog-z.png}}
  \subfigure[z opt] { \label{fig:a}
    \includegraphics[width=0.225\textwidth]{./figures/mtloptW\string_510mode\string_dual\string_p2-iniOpt\string_Z-mtllog-z.png}}
  \subfigure[mtl opt] { \label{fig:a}
  \includegraphics[width=0.225\textwidth]{./figures/mtloptW\string_510mode\string_fixhalf\string_p2-iniOpt\string_Z\string_Lambda\string_updateW-mtllog-z.png}}
  \subfigure[z opt] { \label{fig:a}
    \includegraphics[width=0.225\textwidth]{./figures/mtloptW\string_510mode\string_fixhalf\string_p2-iniOpt\string_Z-mtllog-z.png}}
  \subfigure[mtl opt] { \label{fig:a}
    \includegraphics[width=0.225\textwidth]{./figures/mtlopt-iniOpt\string_Z\string_Lambda\string_updateW-mtllog-z.png}}
  \subfigure[z opt] { \label{fig:a}
    \includegraphics[width=0.225\textwidth]{./figures/mtlopt-iniOpt\string_Z-mtllog-z.png}}
  \caption{Curves of $z$ of the experiments, including the results with material
    optimization (mtl opt) and without material optimization (z opt). Here,
    $m_i$ is the curve for mode $i$.  (a) and (b) are the results of the beam
    with two end swings, (c) and (d) are results of the beam with one end
    swings, while (e) and (f) are the results of the mushroom.}
  \label{result_z}
\end{figure}
% \begin{figure}
%   \centering
%   \subfigure[beam, swing two ends.] { \label{fig:b}
%     \includegraphics[width=0.425\textwidth]{./figures/mtloptW\string_510mode\string_dual\string_p2-iniOpt\string_Z\string_Lambda\string_updateW-mtllog-inner\string_py.png}}
%   \subfigure[beam, swing one end.] { \label{fig:b}
%     \includegraphics[width=0.425\textwidth]{./figures/mtloptW\string_510mode\string_fixhalf\string_p2-iniOpt\string_Z\string_Lambda\string_updateW-mtllog-inner\string_py.png}}
%   \subfigure[mushroom, rotate.] { \label{fig:a}
%     \includegraphics[width=0.425\textwidth]{./figures/mtlopt-iniOpt\string_Z\string_Lambda\string_updateW-mtllog-inner\string_py.png}}
%   \caption{Energy curves of material optimization of the experiments. Subtitles
%     indicate the name of model and the desired animation.}
%   \label{energy_curves}
% \end{figure}

\end{document}
