\documentclass[9pt,twocolumn]{extarticle}

\usepackage[hmargin=0.5in,tmargin=0.5in]{geometry}
\usepackage{amsmath,amssymb}
\usepackage{times}
\usepackage{graphicx}
\usepackage{subfigure}

\usepackage{cleveref}
\usepackage{color}
\newcommand{\TODO}[1]{\textcolor{red}{#1}}

\newcommand{\FPP}[2]{\frac{\partial #1}{\partial #2}}
\newcommand{\argmin}{\operatornamewithlimits{arg\ min}}
\author{Siwang Li}

\title{Material Optimization}

%% document begin here
\begin{document}
\maketitle

\setlength{\parskip}{0.5ex}

\section{Summary}
First, for the optimization of the reduced displacements $z$, Gaussian-Newton
method convergent very fast, but the computational cost for each step is
large. Using Gaussian-Newton method, the computational cost for each step of are
much smaller, but it convergences very slow near the optimal value. We need to
find some effective approach to improve this optimization process.

Second, we make one experiment to check the material optimization with partial
constraints. We found that we can use material optimization to produce some
consistent effects. However, the reduction in the control forces is not large
compare to the results without material optimization, and the convergence speed
is slow.

In the following, we first define our objective function in section
\ref{sec:objective-function}, and then give the alternative optimization scheme
in section \ref{sec:optimization}, and analysis several numerical approaches for
the optimization of the reduced displacements in section
\ref{sec:numer-appr-optim}. Finally, we give the results of material
optimization with partial constraints in section \ref{sec:mater-optim-with}.

\section{Objective  function}\label{sec:objective-function}
Suppose that the resulting animation sequence satisfies the following motion
equation in RS space
\begin{equation} \label{motion_eq}
  \ddot{z} + (\alpha_mI+\alpha_k\Lambda)\dot{z}+\Lambda z = w(t)
\end{equation}
Here, $z$ is the reduced coordinates in RS space, $w(t)$ are the control forces,
while $\alpha_m,\alpha_k,\Lambda$ denotes the elastic materials in the
sub-space. We consider both the keyframe constraints $\{z_i^k,i\in
\mathbb{K}\}$, and partial constraints $\{Cx(z_i,W,x_0)=x_i^c,i\in
\mathbb{C}\}$, where $W$ are the RS-modes, $x_0$ is the rest shape, and
$x(\cdot)$ is the function which maps $z$ to the shape $x$ in full space.

Our target is to optimize both the control force and material to produce an
elastic animation satisfying the constraints, and after discretion
(\ref{motion_eq}) in time, we obtain
\begin{equation}
  \argmin_{z,\Lambda,\alpha_k,\alpha_m}E(z,\Lambda,\alpha_k,\alpha_m)\label{eq:final_eq}
\end{equation}
subject to 
\begin{equation} \label{partial_con}
  Cx(z_i) = {x}_i^c\mbox{, for } i\in \mathbb{C}.
\end{equation}
\begin{equation} \label{constraints}
  z_i = z_i^k \mbox{, for } i\in \mathbb{K}.
\end{equation}
and the objective function $E(\cdot)$ in (\ref{eq:final_eq}) is defined as
\begin{eqnarray}
  E &=& \frac{1}{2}\sum_{i=1}^{T-1}\|\frac{1}{h^2}\hat{z}_i+
  \frac{1}{h}(\alpha_mI+\alpha_k\Lambda)(z_{i+1}-z_{i})+\Lambda
  z_i\|_{2}^2 \nonumber \\
  &=& \frac{1}{2}z^THz  \label{objE}
\end{eqnarray}
where $\hat{z}_i=z_{i+1}-2z_{i}+z_{i-1}$, and $H$ is a sparse matrix, which is
penta-diagonal for each mode, and determined by the materials
$\Lambda,\alpha_k,\alpha_m$ (see the appendix). Here, to simplify the problem,
we only optimize for the displacements $z$ and subspace materials
$\alpha_k,\alpha_m,\Lambda$, and have not optimized for the RS-modes $W$ nor the
rest shape $x_0$ currently.

\section{Optimization}\label{sec:optimization}
Firstly, we apply the partial constraints (\ref{partial_con}) using penalty
method, and apply the keyframe constraints (\ref{constraints}) using variables
replacement(hard constraints). Then, we alternatively optimize the reduced
displacements $z$ (section \ref{sec:optim-reduc-displ}) and material
$\alpha_k,\alpha_m,\Lambda$ (section \ref{sec:optimize-materials}), and we
denote it as the \emph{outer iteration}. And in each outer iteration, we solve
each optimization tasks iteratively, and we call this process as the \emph{inner
  iteration}.

\subsection{Optimize reduced displacements}\label{sec:optim-reduc-displ}
Given $\alpha_k,\alpha_m,\Lambda$, we optimize for $z$ by solving
\begin{equation} \label{optz}
  \argmin_{z}\frac{1}{2}( z^THz+\sum_{i\in \mathbb{C}}\gamma_c\|Cx(z_i)-{x}_i^c\|_{2}^2)
\end{equation}
where $\gamma_c$ is the penalty, and the keyframe constraints (\ref{constraints})
are applied using variables replacement(hard constraints). This is a nonlinear
optimization problem due to $x(z_i)$ appeared in the partial constraints. And we
have tried several different approaches to solve it, as discussed in section
\ref{sec:numer-appr-optim}.

\subsection{Optimize materials}\label{sec:optimize-materials}
Given fixed $z$, we optimize for $\alpha_k,\alpha_m,\Lambda$. In order to find
the optimal material, we need to replace the diagonal matrix $\Lambda$ with a
dense symmetric semi-definite matrix $K=A^TA$, and solve for $\alpha_m,\alpha_k,
A$ using
\begin{equation}\label{opt_k}
  \argmin_{\alpha_k,\alpha_m,A} E(z,A^TA,\alpha_k,\alpha_m)+
  \frac{1}{2}\sum_{i\in \mathbb{C}}\gamma_c\|Cx(z_i)-{x}_i^c\|_{2}^2
\end{equation}
Then we compute $K=A^TA$, and decompose $K=U^T\Lambda'U$, then update $z'=Uz$,
and use the new material $\alpha_k,\alpha_m,\Lambda'$ in section
\ref{sec:optim-reduc-displ} to optimize for the optimal displacements $z$, with
$z'$ as the initial value.

\begin{figure}
  \centering
  \subfigure[Objective Function Value vs. Iterations] { \label{fig:a}
    \includegraphics[width=0.48\textwidth]{./figures/beam-swing-GN.png}
  }
  \subfigure[frame 31] { \label{fig:a}
    \includegraphics[width=0.22\textwidth]{./figures/beam-swing-GN-rlst-f31.png}
  }
  \subfigure[frame 78] { \label{fig:b}
    \includegraphics[width=0.22\textwidth]{./figures/beam-swing-GN-rlst-f78.png}
  }
  \subfigure[frame 128] { \label{fig:a}
    \includegraphics[width=0.22\textwidth]{./figures/beam-swing-GN-rlst-f128.png}
  }
  \subfigure[frame 312] { \label{fig:b}
    \includegraphics[width=0.22\textwidth]{./figures/beam-swing-GN-rlst-f312.png}
  }
  \caption{Convergence of the Gaussian-Newton method for optimizing (\ref{optz})
    in one outer iteration. Pink beams are the results with 3 iterations
    (objective value is $1.1\times 10^6$), and red is the ground truth with 6
    iterations (objective value is $9.8\times 10^5$). Pink and blue points are
    the target positions for the constrained points of the corresponding frames,
    and $T=400$, $r=3$. }
  \label{f_rlst_gn}
\end{figure}

\section{Numerical approaches for optimizing reduced
  displacements}\label{sec:numer-appr-optim}
In the following, we will introduce several different numerical approaches for
optimizing the reduced displacements $z$, e.g problem (\ref{optz}), and discuss
the difficulties of each choice presented in our experiments.

\begin{figure}
  \centering
  \subfigure[Objective Function Value vs. Iterations] { \label{fig:a}
    \includegraphics[width=0.48\textwidth]{./figures/beam-swing-BFGS.png}
  }
  \subfigure[frame 31] { \label{fig:a}
    \includegraphics[width=0.22\textwidth]{./figures/beam-swing-BFGS-GN-rlst-f31.png}
  }
  \subfigure[frame 78] { \label{fig:b}
    \includegraphics[width=0.22\textwidth]{./figures/beam-swing-BFGS-GN-rlst-f78.png}
  }
  \subfigure[frame 128] { \label{fig:a}
    \includegraphics[width=0.22\textwidth]{./figures/beam-swing-BFGS-GN-rlst-f128.png}
  }
  \subfigure[frame 312] { \label{fig:b}
    \includegraphics[width=0.22\textwidth]{./figures/beam-swing-BFGS-GN-rlst-f312.png}
  }
  \caption{Convergence of the BFGS method for optimizing (\ref{optz}) in one
    outer iteration. Red beams are ground truth generated using Gaussian-Newton
    method with 6 iterations (objective value is $9.8\times 10^5$), while green
    beams are the results of BFGS method with $1000$ iterations (objective value
    is $1.68\times 10^6$). Pink and blue points are the target positions for the
    constrained points of the corresponding frames, and $T=400$, $r=3$.}
  \label{f_rlst_bfgs}
\end{figure}

\subsection{Gaussian Newton}\label{sec:gaussian-newton}
This approach usually convergent within $10$ inner iterations for one outer
iteration (see figure \ref{f_rlst_gn}). The disadvantage of this approach is the
computation cost for one inner iteration is high, as we need to solve a large
sparse linear equation for each step.

In detail, let $c(z)=\frac{1}{2}\gamma_c\sum_{i\in
  \mathbb{C}}\|Cx(z_i)-{x}_i^c\|_{2}^2$, then the hessian matrix for the
objective function in (\ref{optz}) is
\begin{equation} \label{hessian}
  H+\gamma_c\frac{\partial^2{c(z)}}{\partial^2{z}}
\end{equation}
When the number of constrained nodes are not large, the computational complexity
for $\frac{\partial^2{c(z)}}{\partial^2{z}}$ is small, which about
$O(|\mathbb{C}|Sr^2)$, where $S$ is the number of cubature samples in our
reduced RS model. As the materials are updated frequently, $H$ should be
recomputed at each outer iteration. The dimension of $H$ is
$(T-|\mathbb{K}|)r\times (T-|\mathbb{K}|)r$, and in practice, we usually require
the number of frames $T$ satisfies $100\le T\le400$, and the dimension $r$ of
the subspace satisfies $30\le r\le80$. Thus for large problem with $T=400$ and
$r=80$, the computational cost for (\ref{optz}) would be expensive.

In our experiments, the outer iteration usually convergent within $30$
iterations. Thus, if we want to solve for (\ref{eq:final_eq}) within one second,
we need to solve (\ref{optz}) within $1/30$ second. If there is no material
optimization, $H$ is constant all the time, we can pre-compute $H^{-1}$, and use
Lagrange method to apply the partial constraints, then it is even possible to
solve (\ref{optz}) within $1/100$ second using Green functions. However, for our
problem, as $H$ is not constant (updated at each outer iteration), it is much
more challenge to solve (\ref{optz}) efficiently.

\subsection{BFGS}\label{sec:bfgs}
Using this approach, we only need to compute the gradient of the objective
function in (\ref{optz}), thus the computational cost for each step in the inner
iteration is much cheaper than Gaussian Newton method. However, it convergences
very slow. As shown in figure \ref{f_rlst_bfgs}, in the first 30 iterations, it
convergent very fast, however when it is near the optimal results, it is hard to
convergent. Even after 1000 iterations, the resulting animation are differs
greatly from the ground truth results generated using the Gaussian-Newton
method. As the computational cost for one step is cheap for this method, we can
consider to use it to produce the initial value for the Gaussian-Newton method.

% \subsection{BFGS with adjoint method}
% Let's denote $q_i = (v_i,z_i)$ as the status of frame $i$, and assume that the
% integration function for equation (\ref{motion_eq}) is
% \begin{equation} \label{integ}
%   q_{i+1} = f_i(q_i,w_i)
% \end{equation}
% Then we can rewrite the optimization problem (\ref{optz}) as
% \begin{equation}\label{adj_final_eq}
%   \argmin_{w}\frac{1}{2}\sum_{i=0}^{T-2}\|w_i\|_2^2+\sum_{i\in
%     \mathbb{C}}\|Cx(z_i)-{x}_i^c\|_{M_c}^2+\|z_i-z_i^k\|_{M_k}^2
% \end{equation}
% subject to
% \begin{equation} \label{adj_con}
%   q_{i+1} = f_i(q_i,w_i)
% \end{equation}
% Here, we use penalty method to apply both the partial and keyframe constraints,
% and optimize for the control forces $w$ directly instead of $z$. we apply the
% motion equations as hard constraints according to (\ref{adj_con}). Then, we use
% adjoint method to compute the gradient of this problem, and use BFGS method to
% solve it. When $w$ is obtained, we compute $z$ using the integration method
% (\ref{adj_con}), and use it to further optimize for the materials.

% As the motion equations for each mode in (\ref{adj_con}) are decoupled, the
% computational cost for each step in the inner iteration is much cheaper than
% computing general Gaussian Newton method in section ~\ref{sec:gaussian-newton},
% but more expensive than the approach in section \ref{sec:bfgs}. And this
% approach also convergent much slower than Gaussian Newton method. In our
% experiments, it takes more than $200$ inner iterations to convergence for one
% outer iteration.

\subsection{Conclusion of the numerical approaches}
We have tried several different method to optimize for (\ref{optz}), and found
that 
\begin{itemize}
\item The Gaussian Newton method convergent very fast, but the computational
  cost for each step is large.
\item The BFGS method convergent fast at the beginning, but become very slow
  near the optimal results.
% \item Combine adjoint method with the BFGS method, the computational cost for
%   each step smaller than Gaussian-Newton method, and more expensive than
%   computing the gradient directly(~\ref{sec:bfgs}). And the convergence speed is
%   still very slow compare to Gaussian-Newton method.
\end{itemize}
Thus, one possible numerical choice is to improve speed for each step of the
Gaussian Newton approach by using muti-grid method on the decoupled modes. For
example, we can first optimize for the first $0,1,2$ modes, then $3,4,5$, and so
on. For each sub-problem, the computational cost is much smaller. And if this
multigrid method can convergent fast, we can significantly improve the
optimization speed. Further more, we can use the BFGS method to produce the
initial value for the Gaussian-Newton method.

\section{Material optimization with partial constraints}\label{sec:mater-optim-with}
In this section, we present the results of the material optimization with
partial constraints, e.g the optimization results of (\ref{eq:final_eq}),
(\ref{partial_con}) and (\ref{constraints}). We apply four position constraints
at frame 78 and 128 respectively, and constrained the displacements at frame
$0,1,2$ as zero using keyframe constraints. We use Gaussian-Newton method to
optimize both (\ref{optz}) and (\ref{opt_k}), and the convergence curve of outer
iteration is shown in figure \ref{f_rlst_parcon}, while the resulting animation
can be found at the attached video. This experiment shows that,
\begin{itemize}
\item By using material optimization, we success to find a mode which represents
  the cycling motion satisfying the constraints with minimal control forces, and
  produce consistent effects.
\item The outer iteration convergent slow. Though the objective value obtained
  at iteration 30 is similar to the objective value in iteration $50$, the
  resulting animations differ greatly.
\item The reduction of the control forces in this example is not large. Without
  material optimization, the norm of the control forces is $9.8\times 10^5$, and
  it reduces to $7.9\times 10^5$ by using material optimization.
\end{itemize}
\begin{figure}
  \centering
  \subfigure[Objective Function Value vs. Outer iterations] { \label{fig:a}
    \includegraphics[width=0.38\textwidth]{./figures/beam-swing-BFGS-pcon.png}
  }
  \subfigure[$z$ with no material optimization] { \label{fig:a}
    \includegraphics[width=0.38\textwidth]{./figures/beam-swing-BFGS-pcon-no-mtl.png}
  }  
  \subfigure[$z$ of 30 iterations] { \label{fig:a}
    \includegraphics[width=0.38\textwidth]{./figures/beam-swing-BFGS-pcon-Uz_30it.png}
  }
  \subfigure[$z$ of 50 iterations] { \label{fig:b}
    \includegraphics[width=0.38\textwidth]{./figures/beam-swing-BFGS-pcon-Uz_50it.png}
  }
  \subfigure[frame 100] { \label{fig:a}
    \includegraphics[width=0.22\textwidth]{./figures/beam-swing-BFGS-pcon-comp_f100.png}
  }
  \subfigure[frame 399] { \label{fig:b}
    \includegraphics[width=0.22\textwidth]{./figures/beam-swing-BFGS-pcon-comp_f399.png}
  }
  \caption{Results of material optimization with partial constraints. (a) the
    convergence curve of the outer iterations. Before optimization, the
    objective value is $6.1\times 10^9$, which is not shown in the figure. (b)
    the resulting $z$ without material optimization. (c) the resulting $z$ of
    each mode after material optimization with 30 iterations. (d) $z$ with 50
    iterations. (e) and (f) are two frames of the resulting animation, where
    green are the results of 30 iterations, while blue are the results of 50
    iterations. Here, $T=400$, and $r=3$.}
  \label{f_rlst_parcon}
\end{figure}

\appendix
\section{Structure of $H$}
The hessian $H$ of (\ref{objE}) is a sparse matrix, and it is penta-diagonal for
each mode. We consider one mode $j$, and let
\begin{eqnarray}\label{abc}
  a &=& \frac{1}{h^2}+\frac{1}{h}(\alpha_m+\alpha_k\Lambda_{jj})\\
  b &=& \Lambda_{jj}-\frac{1}{h}(\alpha_m+\alpha_k\Lambda_{jj})-\frac{2}{h^2}\\
  c &=& \frac{1}{h^2}
\end{eqnarray}
then
\begin{equation} \label{}
  E(z^j) = \frac{1}{2}\sum_{i=1}^{T-2}\|az^j_{i+1}+bz^j_{i}+cz^j_{i-1}\|_2^2
\end{equation}
As an example, $H$ of one mode with $T=6$ is
\[
\left( \begin{array}{ccccccc}
    c^2&bc&ac&0&0&0\\
    bc&c^2+b^2&bc+ab&ac&0&0\\
    ac&bc+ab&c^2+b^2+a^2&bc+ab&ac&0\\
    0&ac&bc+ab&c^2+b^2+a^2&bc+ab&ac\\
    0&0&ac&bc+ab&b^2+a^2&ab\\
    0&0&0&ac&ab&a^2\\
  \end{array} \right)
\]
It is clear that for each mode, $H$ is a symmetric penta-diagonal matrix.

% \section{Example of one outer iteration}
% We demonstrate the details for one outer iteration here. Suppose that the at the
% beginning of the $k$-th outer iteration we obtain the optimal reduced
% coordinates $z^{k}\in R^{r\times T}$, and current RS-modes is $W^{k}\in
% R^{n\times r}$. Now we optimize for the materials in the subspace, and obtain
% $\alpha_m^{k+1},\alpha_k^{k+1},K^{k+1}$. Then we decompose $K^{k+1}=U^T\Lambda'
% U$, and let $z^k=Uz'$

\end{document}