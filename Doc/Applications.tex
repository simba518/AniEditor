\documentclass[9pt]{extarticle}

\usepackage[hmargin=0.5in,tmargin=0.5in]{geometry}
\usepackage{amsmath,amssymb}
\usepackage{times}

\usepackage{cleveref}
\usepackage{color}
\usepackage{longtable}
\newcommand{\TODO}[1]{\textcolor{red}{#1}}

\newcommand{\FPP}[2]{\frac{\partial #1}{\partial #2}}
\newcommand{\argmin}{\operatornamewithlimits{arg\ min}}
\author{Siwang Li}

\title{Possible Applications of this approach}

%% document begin here
\begin{document}
\maketitle

\setlength{\parskip}{0.5ex}

\section{Applications}
\begin{itemize}
\item Animation design with partial and keyframe constraints: much smaller
  control forces, natural results even with unsuitable initial material,
  consistent effects, tight partial constraints, large deformation, more compact
  subspace, and the optimized material can be used in the following editing.
\item Animation editing: interactive performance, produce consistent
  periodically motions, support animations produced by hand or simulated
  results, support local editing, and produce localized mode.
\item Recover the elastic material: simulated results or even animation sequence
  obtained from the real world can be used as input.
\item Generate a compact subspace to represent the resulting animation.
\item Adjust the motion of different objects, or the different parts of an
  object.
\item Static deformation, efficient shape design.
\item Produce local deformation modes during the editing, and we can control
  the frequency of each mode independently.
\end{itemize}

\section{Examples}
\begin{center}
    \begin{longtable}{ | p{1.7cm} | p{8.0cm} | p{8.0cm}|}
    \hline
    Model & Action & highlight \\ \hline 

    flower& simulate a sequence with secondary dynamics, edit it to move around
    a circle. & 1. trace prescribed path. 2. keep input details. \\ \hline

    bird & open the mouth, open the mouth in such a way that the beak should
    looks rigid. & 1. material study, 2. constraint a portion of the object.\\
    \hline

    dinosaur & given simulated results, recover the material. & material recover. \\ \hline

    beam & given simulated results, recover the material. & material recover. \\ \hline

    bird & swing the wings. & keyframe constraints, interactive editing. \\
    \hline

    beam and bird & transfer the motion from the bird to the beam. & motion
    transfer \\ \hline

    dinosaur & jumping or walking. & complicated results, including keyframe,
    partial, material optimization, and motion transfer. \\ \hline

    bottle & example based material. & example based material.  \\ \hline

    monster & monster hit by a ball, edit it to response to the collision. &
    handle collision by precise positional constraints.  \\ \hline

    % bird & 1. swing the wings symmetrically. 2. open the mouth in such a way
    % that the beak should looks rigid. 3. adjust the motion of the head and the
    % wings. 3. compare the results with no material optimization. 4. adjust the
    % frequency of the motion of wings and beak independently. & 1. constraint a
    % portion of the object.  2. consistent effects.  3. avoid the artifacts due
    % to unsuitable initial material.  4. interactive performance.
    % 5. localized deformation. \\
    % \hline

    % flower uniform& 1. move periodically with different path: circle, 8 or
    % wave. 2. adjust the motion of different parts. 3. compare the results with
    % no material optimization.& 1. consistent periodically motions. 2. contract
    % motion pattern. 3. trace prescribed path.\\ \hline

    % dinosaur& walking or swing the tail.& 1. static deformation. 2. study the material. 3. material recover. \\ \hline

    % flower non-uniform& 1. recover the material from simulated results. 2. compare the
    % results with no material optimization.& elastic material recover.\\
    % \hline

    % flowers & many flowers move simultaneously to produce some pattern. & adjust
    % the motion of different objects.\\
    % \hline

    % Character I &use keyframes as example to design material, then interactive
    % edit the resulting animation with partial spacetime constraints. & 1. use
    % keyframes as examples to design the elastic material, and use it in the
    % spacetime optimal control framework. 2. interactive editing.\\
    % \hline

    % octopus & 1. jump with all legs move simultaneously. 2. large local
    % deformation at the end of the legs. & 1. partial constraints. 2. large
    % deformation. 3. adjust the motion of different parts.\\ \hline

    % beam &1. swing two ends simultaneously. 2. swing one half with another half
    % fixed. 3. interactive editing with the optimized material and
    % modes. 4. material recover. 5. static deformation.& present all the
    % advantages of our method.\\\hline

    % & &animation editing with simulated results. \\ \hline

    % simple plant& &material recovered from real world. \\ \hline

    \end{longtable}
\end{center}

\section{Pipeline}
\begin{center}
    \begin{longtable}{ | p{3.0cm} | p{12.0cm} | p{2.0cm}|}
    \hline
    Task & Todo & People \\ \hline

    design the examples & 1. show and discuss the examples with others. 2. ask
    others to prepare the obj meshes.& zhou, yang.\\ \hline

    prepare input data& teach zhang to use
    the script to produce all the data.& zhang.\\ \hline
    
    live demo & 1. operate the bird to produce some results. 2. test the
    record-replay operation. 3. teach zhang to produce other results. 4. screen
    recording. 5. ask qiu to improve rendering. & zhang, qiu.\\ \hline

    keyframes, partial & 1. write an application or script to
    convert the keyframes into partial constraints. 2. produce some results
    using keyframes. 3, teach zhang to do it.& zhang.\\ \hline

    material recover &1. produce simulated results using reduced and full stvk
    method. 2. write a script to do all automatically. 3. obtain the animation
    sequence from real world. & \\ \hline

    rendering &1. build a pipeline for rendering. 2. teach zhang to use it. & zhang.\\ \hline

    demo, figures and table & 1. design the demo, figures and performance
    table. 2. write script to produce figures and performance statistic
    automatically. 2. ask zhang to compile the data in another computer.& zhang\\
    \hline

    

    \end{longtable}
\end{center}

\end{document}

