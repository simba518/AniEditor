\documentclass[9pt,twocolumn]{extarticle}

\usepackage[hmargin=0.5in,tmargin=0.5in]{geometry}
\usepackage{amsmath,amssymb}
\usepackage{times}
\usepackage{graphicx}
\usepackage{subfigure}

\usepackage{cleveref}
\usepackage{color}
\newcommand{\TODO}[1]{\textcolor{red}{#1}}

\newcommand{\FPP}[2]{\frac{\partial #1}{\partial #2}}
\newcommand{\argmin}{\operatornamewithlimits{arg\ min}}
\author{Siwang Li}

\title{Experiment of Optimization of the Basis}

%% document begin here
\begin{document}
\maketitle

\setlength{\parskip}{0.5ex}

\section{First experiment}
\subsection{Setup}
Expand the RS basis $\hat{W}$ as
\begin{equation} \label{w}
  \hat{W}' = \left( \begin{array}{ccc}
      \hat{W} & y_{k0} &y_{k1} \cdots
    \end{array} \right)
\end{equation}
and then using the new eigenvalues as
\begin{equation} \label{e}
  \Lambda' = \left( \begin{array}{cccc}
      \Lambda &\\
      &\lambda_{k0}&\\
      &&\lambda_{k1}&\\
      &&&\ddots
    \end{array} \right)
\end{equation}
where $y_{ki}$ is the RS coordinates of the keyframes $u_{ki}$, while
$\lambda_{ki}$ is some given eigenvalues with each these keyframes.

We hope that 
\begin{itemize}
\item the $u_{ki}$ can be exactly approximated by using the modes in $\hat{W}'$,
\item with material optimization, we can find only a few modes to represent the
  final motion, and
\item the motion is smooth, i.e, no motion with high frequency.
\end{itemize}

\subsection{Results}
$u_{ki}$ can be well approximated by using $\hat{W}'$, and the control forces
are much smaller by using material optimization, and the motion is
smooth. However, we can not find one single mode to represent the motion. This
is because in our setup, we assume that the modes $y_{ki}$ are decoupled, and
there is only one keyframe for each mode, and the initial energy is very small.

\begin{figure}
  \centering

  \subfigure[Inner iteration.] { \label{fig:a}
  \includegraphics[width=0.4\textwidth]{./figures/mtlopt_cen_keyWOpt_Z_AtA-inner_py.png}}

  \subfigure[Outer iteration.] { \label{fig:b}
  \includegraphics[width=0.4\textwidth]{./figures/mtlopt_cen_keyWOpt_Z_AtA-outer_py.png}}

  \subfigure[z curve.] { \label{fig:c}
  \includegraphics[width=0.4\textwidth]{./figures/mtlopt_cen_keyW_Opt_Z_AtA_curveZ.png}}

  \caption{Results of the first experiment. (a) and (b) Energy-iteration
    curves. (c) curves for $z$.}
  \label{opt-6}
\end{figure}

\section{Second experiment}
\subsection{Setup}
We expand compute $\hat{W}'$ and $z_k$ using the following method 
\begin{eqnarray*}
  z_{k} &=& \argmin_{z_{k}} \frac{1}{2}\|\hat{W}z_k-y_k\|_2^2   \\
  y_k' &=& y_k - \hat{W}z_k\\
  \hat{W}' &=& \left( \begin{array}{ccc}
      \hat{W} &y_k'
    \end{array} \right)\\
  z_k' &=& \left( \begin{array}{ccc}
      z_k & 1
    \end{array} \right)\\
  \Lambda' &=& \left( \begin{array}{ccc}
      \Lambda & \\
      & \lambda_0
    \end{array} \right)
\end{eqnarray*}
where we choose $\lambda_0$ as the first element of $\Lambda$. We hope to find
one single mode to represent the motion, and the whole motion is smooth.
\subsection{Results}
The results of the this experiments are shown in figure \ref{exp2}. It can be
seen that the resulting shape(yellow) is non-smooth. And this is because the
additional mode 3(red) is non-smooth. Thus, we can not simply expand the
RS-modes $\hat{W}$ as above.

\begin{figure}
  \centering
  \includegraphics[width=0.48\textwidth]{./figures/mtlopt_cen_keyW_Opt_Z_AtA_curveZ_exp2.png}
  \caption{Results of the second experiment. Yellow is the resulting shape, blue
  is mode 0, green is mode 2, and red is mode 3.}
  \label{exp2}
\end{figure}

\end{document}