\documentclass[twocolumn,a4paper]{article}
\usepackage{graphicx}
\usepackage{amsfonts}
\usepackage{color}
\usepackage{lineno}
\setlength{\columnsep}{10pt}
\setlength{\oddsidemargin}{0pt}
\setlength{\topmargin}{0pt}
\setlength{\headheight}{0pt}
\setlength{\headsep}{0pt}
\setlength{\marginparsep}{0pt}
\setlength{\marginparwidth}{0pt}
\addtolength{\voffset}{-50pt}
\setlength{\textheight}{26.8cm}
\author{Siwang Li}

\title{Implementation of Material Optimization}

%% document begin here
\begin{document}
\maketitle

\section{Input Animation}
The input are some displacements $P = (p_1,\cdots,p_T) \in R^{3n\times T}$ of a
beam with respect to the rest shape $r$. We use implicit integrator, zero
damping ($\alpha_k=\alpha_m=0$), and same time step $h=0.1$ in all the
experiments. Some of the animation are produced with some nodes are fixed, but
no other external forces. We produce such results by setting a deformed shape as
initial state, then record the simulated result. And for the reduced simulator,
constraints are achieved by removing the corresponding rows from the basis
matrix.

\section{PCA on RS coordinates}
Firstly, we compute the RS coordinates $\hat{P} =
(\hat{p}_1,\cdots,\hat{p}_T)$.  Then we solve the SVD problem $\hat{P} = U\Sigma
V^T$, and use the first $m$ columns of $U$ to obtain $W' = (u_1,\cdots,u_m)$,
and the corresponding reduced coordinates $z' = ((\Sigma V^T)_1,\cdots,(\Sigma
V^T)_T)$.

\subsection{Experiments}
\begin{itemize}
\item Check the residual $\hat{P}-W'z'$ to make sure the PCA is correct. It
  should be very small.({\color{red}$<10^{-4}$})
\item Use the 6-th mode $w_6$ to produce reduced coordinates $z_i=i\times w_6$,
  and corresponding RS coordinates $\hat{p}_i=\hat{W}z_i$. Then use $\hat{P}$ to
  compute $W'$ and $z'$. Then check the residual $\|(|w_6|z)-z'\|_2^2$ and
  $\|\frac{\hat{W}}{|w_6|}-W'\|_2^2$({\color{red}$W'$ is normalized and
    $\hat{W}$ not. The residuals are small than $10^{-16}$}).
\item Use both the 6-th and 7-th modes to make the above
  experiment.({\color{red}Residual is large ($>0.2$), as columns in $W'$ are
    orthogonal, while $\hat{W}$ not.})
\item Use the first 20 modes to simulate an animation using a RS simulator with
  some nodes are constrained, and record $z$ and $\hat{P}$. Then check the
  residual $\|z-z'\|_2^2$ and $\|\hat{W}-W'\|_2^2$.({\color{red} have not made.})
\item Generate an animation using a Reduced StVK simulator with some nodes are
  constrained. Then use it to compute $W'$ and $z'$. Then check the residual
  $\|z-z'\|_2^2$ and $\|\hat{W}-W'\|_2^2$.({\color{red} have not made.})
\end{itemize}

\subsection{Conclusion}
With PCA, the $W'z'$ can well approximate the input RS coordinates. However, the
residual $\|\hat{W}-W'\|_2^2$ and $\|z-z'\|$ can be very large. This is because
the PCA result $W'$ are orthogonal, while the original modal matrix $\hat{W}$ is
usually not.

\section{Approximate $K$ Without Constraints}
To compute $K$, we discrete $z(t)$ in time, and define a quadratic energy
function based on implicit integration
\begin{equation} \label{qua-en}
  E(k) = \sum_{i=2}^{T-1} \|\frac{1}{h^2}(z_{i+1}-2z_{i}+z_{i-1}) + K(k)z_i\|_2^2
\end{equation}
where $K(k)$ is a symmetric matrix which is obtained by properly assemble $k_i$
into $K$. This quadratic energy function can be minimized by solving
\begin{equation} \label{sol}
  Hk = -b
\end{equation}
where $H = \frac{\partial^2{E}}{\partial^2{k}}$ is a constant matrix, and $b =
\frac{\partial{E}}{\partial{k}}|_{k=0}$ is a constant vector. 

\subsection{Experiments}
\begin{itemize}
\item Use an random vector $x$ to check $E(x)-(\frac{1}{2}x^THx+b^Tx+E(0))$ to
  make sure $H$ and $b$ are correct.
\item Check $Hk = -b$ to make sure the equation solving is correct.
\item We use the 6-th mode to produce the animation $z'$ with a RS simulator
  without constraints, then use it to produce a $K\in R^{1\times1}$. Check the
  residual $|K-\lambda_6|$, and $E(k)$.
\item Use different time step $h$ to make the above experiment.
\item We use 20 modes from $6$ to $26$ to produce the animation $z'$ with a RS
  simulator without constraints, then use it to produce a $K$. Check the
  residual $\|K-diag(\lambda_6,\cdots,\lambda_{26})\|_2^2$, and $E(k)$.
\item Use different time step $h$ to make the above experiment.
\item We use a full StVK simulator to produce an animation $P =
  (p_1,\cdots,p_T)$ without constraints. Then compute $K$ and check the residual
  $E(k)$.
\item Use different time step $h$ to make the above experiment.
\item Use the same $P$ as above, but compute $z' = \hat{W}^{-1}\hat{P}$. Check
  the residual $E(\Lambda)$.
\item Use same $z'$ as above, compute $K$, check the residual $E(k)$.
\item Produce all the input and out animation, check the deformation results and
  dynamics.
\end{itemize}

\section{Approximate $K$ with Constraints}


% As $H$ may be rank
% deficient, we make a compact SVD $H=U_1\Sigma_1 V^T_1$, and compute
% $k=V_1\Sigma_1^{*}U_1^T(-b)$, where $\Sigma_1^{*}$ is the pseudo inverse of
% $\Sigma_1$. {\color{red}Experiment shows that the residual $Hk+b$ is small (less than
% $10^{-7}$), however $E(k)$ is very large (larger than $10^5$).}



%% references
% \begin{thebibliography}{99}
% \bibitem{sig2011} Fast simulation of skeleton-driven deformable body
%   characters.
% \end{thebibliography}

\end{document}