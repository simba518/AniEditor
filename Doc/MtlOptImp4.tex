\documentclass[9pt,twocolumn]{extarticle}

\usepackage[hmargin=0.5in,tmargin=0.5in]{geometry}
\usepackage{amsmath,amssymb}
\usepackage{times}

\usepackage{cleveref}
\usepackage{color}
\newcommand{\TODO}[1]{\textcolor{red}{#1}}

\newcommand{\FPP}[2]{\frac{\partial #1}{\partial #2}}
\newcommand{\argmin}{\operatornamewithlimits{arg\ min}}
\author{Siwang Li}

\title{Elastic Material Optimization}

%% document begin here
\begin{document}
\maketitle

\setlength{\parskip}{0.5ex}

\section{Optimize Reduced Stiffness Material}
In the first few experiments, we intend to optimize only the stiffness materials
in subspace.

\subsection{Input data}
First, we need to produce an animation sequence $z'_0,\cdots,z'_{T}$ in subspace
without external forces. We fix one end of a beam, and make modal analysis to
obtain a reduced elastic system
\begin{equation} \label{mq}
  \ddot{z}' + (\alpha_m+\alpha_k \Lambda')\dot{z}' + z' = f
\end{equation} 
Then we deform the beam using the first and sixth modes to obtain the initial
state $z'_0$ of a reduced simulator. Finally, we produce the remaining frames
$z'_1,\cdots,z'_{T}$ using this simulator with $f=0$. 

Second, we select several frames $z'_{0},z'_{k1},\cdots,z'_{k_n}$ from the above
sequence, and use them as keyframes in the following experiments.

\subsection{Experiments}
In all of these experiments, we use $z'_{0},z'_{k1},\cdots,z'_{k_n}$ as keyframes,
and then solve for a new sequence $z_0,\cdots,z_{T}$ by solving this
space-time constraint problem
\begin{eqnarray} \label{sp}
  \min_{z,\Lambda} E(z,\Lambda) &=& \frac{1}{2} \int_{0}^{T}\|\ddot{z} +
  (\alpha_m+\alpha_k \Lambda)\dot{z} + z\|_2^2\\
  \mbox{s.t. } z_{k_i} &=& z'_{k_i}
\end{eqnarray}
We hope that $z_i$ can satisfy these keyframes. What's
more, we also hope that we can find some proper elastic material $\Lambda$ to
reduce the control forces. 

\subsubsection{Optimize $z$ only}
In this experiment, we only optimize for $z$. We first fix the elastic material
$\Lambda = \Lambda'$, and hope that the control forces are nearly zero. And then
use a different material $\Lambda=\Lambda''$ to optimize $z$, and it should
produce much larger control forces. Specially, we need to calculate the control
forces when $\Lambda= 0$.

\subsubsection{Optimize $z$ and $\Lambda$}
We use $\Lambda=0$ as the initial value to optimize equation (\ref{sp}), and
hope that we can find $\Lambda=\Lambda'$ (maybe only for the modes which are
used to generated the keyframes).

\subsubsection{Change time step and damping}
We use time step and damping that are different from those used to generate the
input animation, and make the second experiment. We hope that the control forces
are much smaller, and $\Lambda$ convergent to some reasonable values other
than zero.

\subsubsection{Use dense stiffness matrix $K$}
In this experiment, we use a dense matrix $K=A^TA$ instead of $\Lambda$ in
equation (\ref{sp}), and then optimize for both $A$ and $z$. Then we take the
eigen decomposition $K = U^T\Lambda U$, and we hope that $\Lambda$ is close to
the reference value $\Lambda'$.

Then we will use different time step and damping to make the same experiment,
and we hope that $\Lambda$ convergent to some reasonable values other than zero,
and the control forces are smaller than use $\Lambda=\Lambda'$.

\subsubsection{Exchange the keyframes}
In this experiment, we change the order of the keyframes. In such case, if we
fix $\Lambda = \Lambda'$, and solve for $z$, the control forces will much larger
than zero, and this is the general case in elastic optimization. Then we
optimize for $K=A^TA$ and $z$ as above experiment, and we hope that we can find
some better elastic materials for this animation, i.e, smaller control forces
and reasonable $\Lambda$.

\section{More experiments}
\subsection{Optimize Basis $\hat{W}$}
If we can obtain the desired results in the above experiments, we can try to
introduce $\hat{W}$ as variables in the optimization process, and use the
"Material Estimation" method in our proposal to solve this optimization
problem. And in this experiment, we can try to use some keyframes that can not
be defined by the initial $\hat{W}$, and we hope that the optimized $\hat{W}$
can well define these shapes.

\subsection{Partial constraints}
We can try to use some partial constraints $Cz = u^c$ instead of the keyframes,
and then solve for the optimal material.


%% references
% \begin{thebibliography}{99}
% \bibitem{sig2011} Fast simulation of skeleton-driven deformable body
%   characters.
% \end{thebibliography}

\end{document}