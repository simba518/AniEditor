\documentclass[9pt,twocolumn]{extarticle}

\usepackage[hmargin=0.5in,tmargin=0.5in]{geometry}
\usepackage{amsmath,amssymb}
\usepackage{times}

\usepackage{cleveref}
\usepackage{color}
\newcommand{\TODO}[1]{\textcolor{red}{#1}}
\newcommand{\RED}[1]{\textcolor{red}{#1}}

\newcommand{\FPP}[2]{\frac{\partial #1}{\partial #2}}
\newcommand{\argmin}{\operatornamewithlimits{arg\ min}}
\author{Siwang Li}

\title{Elastic Material Optimization}

%% document begin here
\begin{document}
\maketitle

\setlength{\parskip}{0.5ex}

\section{Optimize Reduced Stiffness Material}\label{sec:optim-reduc-stiffn}
In the first few experiments, we intend to optimize only the stiffness materials
in subspace.

\subsection{Input data}
First, we need to produce an animation sequence $z'_0,\cdots,z'_{T}$ in subspace
without external forces. We fix one end of a beam, and make modal analysis to
obtain a reduced elastic system
\begin{equation} \label{mq}
  \ddot{z}' + D'\dot{z}' + \Lambda'z' = f
\end{equation} 
Then we deform the beam using the first few modes to obtain the initial state
$z'_0$ of a reduced simulator. Finally, we produce the remaining frames
$z'_1,\cdots,z'_{T}$ using this simulator with $f=0$.

Second, we select several frames $z'_{0},z'_{k1},\cdots,z'_{k_n}$ from the above
sequence, and use them as keyframes in the following experiments.

\subsection{Experiments}
In all of these experiments, we use $z'_{0},z'_{k1},\cdots,z'_{k_n}$ as keyframes,
and then solve for a new sequence $z_0,\cdots,z_{T}$ by solving this
space-time constraint problem
\begin{eqnarray} \label{sp}
  \min_{z,\Lambda} E(z,\Lambda) &=& \frac{1}{2} \int_{0}^{T}\|\ddot{z} +
  D\dot{z} + \Lambda z\|_2^2\\
  \mbox{s.t. } z_{k_i} &=& z'_{k_i}
\end{eqnarray}
And in our implementation, we discrete this problem using Central Difference
Method
\begin{equation}\label{spd}
  E(z) = \sum_{i=2}^{T-1} \|\frac{1}{h^2}\hat{z}_i+\frac{1}{h}D(z_{i+1}-z_{i})+
  \Lambda z_i\|_2^2
\end{equation}
where $h$ is the time step, and $\hat{z}_i=z_{i+1}-2z_{i}+z_{i-1}$.  We hope
that the resulting $z_i$ can satisfy these keyframes, and we can find some
proper elastic material $\Lambda$ to reduce the control forces.

\subsubsection{Optimize $z$ only}
In this experiment, we only optimize for $z$. We first fix the elastic material
$\Lambda = \Lambda'$, and hope that the control forces are nearly zero. And then
use a different material $\Lambda=\Lambda''$ to optimize $z$, and it should
produce much larger control forces. Specially, we need to calculate the control
forces when $\Lambda= 0$.

\subsubsection{Optimize $z$ and $\Lambda$}
We use $\Lambda=0$ as the initial value to optimize equation (\ref{sp}), and
hope that we can find $\Lambda=\Lambda'$ (maybe only for the modes which are
used to generated the keyframes).

\subsubsection{Change time step and damping}
We use time step and damping that are different from those used to generate the
input animation, and make the second experiment. We hope that the control forces
are much smaller, and $\Lambda$ convergent to some reasonable values other
than zero.

\subsubsection{Use dense stiffness matrix $K$}
In this experiment, we use a dense matrix $K=A^TA$ instead of $\Lambda$ in
equation (\ref{sp}), and then optimize for both $A$ and $z$. Then we take the
eigen decomposition $K = U^T\Lambda U$, and we hope that $\Lambda$ is close to
the reference value $\Lambda'$.

Then we will use different time step and damping to make the same experiment,
and we hope that $\Lambda$ convergent to some reasonable values other than zero,
and the control forces are smaller than use $\Lambda=\Lambda'$.

\subsubsection{Exchange the keyframes}\label{sec:exchange-keyframes}
In this experiment, we change the order of the keyframes. In such case, if we
fix $\Lambda = \Lambda'$, and solve for $z$, the control forces will much larger
than zero, and this is the general case in elastic optimization. Then we
optimize for $K=A^TA$ and $z$ as above experiment, and we hope that we can find
some better elastic materials for this animation, i.e, smaller control forces
and reasonable $\Lambda$.

\begin{center}
  \begin{table*}[ht]
    {\small
      \hfill{}
      \begin{tabular}{ | l | c | c | c | c | c | c | c| c | c | c | c | c |}
        \hline
        modes& \multicolumn{5}{c|}{Key frame unchanged: 0,\RED{12,50},85,140,199} & 
        \multicolumn{3}{c|}{Key frame changed: 0,\RED{50,12},85,140,199}\\\cline{2-9}
        &$E(z)$ & $E(z,D)$ & $E(z,D,\Lambda)$ & $E'(z,D,\Lambda)$& $E'(z,D,A)$ &
        $E(z,D)$& $E'(z,D,\Lambda)$& $E'(z,D,A)$\\ \hline
        0 &29556&$1.3\times10^{-15}$&$\RED{1.8\times10^{8}}$&$6.1\times10^{-16}$&&$3.0\times10^{6}$&$2.98\times10^{6}$&\\ \hline
        1 &199176&$1.1\times10^{-16}$&  &  &  &  & &\\ \hline
        2 & &$1.4\times10^{-17}$&  &  &  &  & &\\ \hline
        0,1 &&$1.7\times10^{-14}$&  &$7.1\times10^{-16}$&$4.5\times10^{-5}$&$3.2\times10^{6}$&$3.0\times10^{6}$&$\RED{2.9\times10^{6}}$\\ \hline
        0,1,2 & &\RED{346285}&  & &$2.6\times10^{-5}$  &  &$3.0\times10^{6}$&$2.9\times10^{6}$\\ \hline
      \end{tabular}
      \hfill{}
}
\caption{Experiment results for section \ref{sec:optim-reduc-stiffn}. $E(x)$ is
  the residual when we optimize for $x$, and we use 0 as the initial value. 
  While for $E'(x)$, we use $z=0.8z'$ as the initial value. For the last three columns, 
  we changed the order of two keyframes (frame 50 and 12). }
\label{rlst}
\end{table*}
\end{center}

\subsection{Results and analysis}
The experiments results are shown in table \ref{rlst}. The parameters for
generating the input data are, eigen values $\Lambda' =
(0.0497447,0.15304,1.61794)$, time step $h=0.3$, damping $\alpha_k'=\alpha_m'
=0.001$, and initial value $z'_0 = (10000,7000,7000)$. And we further required
that \RED{$\Lambda_i \ge 0$}, $D$ is a diagonal matrix with \RED{$D_{ii}\ge 0$},
and \RED{$A_{ij} \ge 0$}.

From the results in table \ref{rlst}, we can conclude that
\begin{itemize}
\item The residual $E(z)$ is very large, and it can be reduced by introducing
  the damping $D$ into the optimization, e.g. use $E(z,D)$. This is because the
  implicit integration method for generating $z'$ will introduce numerical
  damping, while the central difference method in equation (\ref{spd}) will not.
\item We need good initial value to optimize $z,D,\Lambda$ simultaneously. See
  the difference between $E(z,D,\Lambda)$ and $E'(z,D,\Lambda)$ of mode 0.
\item With good initial value, we can find a dense $K=A^TA$ with small residual
  $E'(z,D,A)$ (see $E'(z,D,A)$ of modes (0,1) and (0,1,2) with the keyframe unchanged.).
\item When we change the order of the keyframes as described in section
  \ref{sec:exchange-keyframes}, $E'(z,D,A)$ is smaller than $E'(z,D,\Lambda)$
  and $E(z,D)$, as shown by the results of modes (0,1) with the keyframe
  changed. \RED{However, the difference is small, and $E'(z,D,A)$ takes more
    than 2000 iterations to convergent, while $E'(z,D,\Lambda)$ takes only 70
    iterations.}
\item \RED{The residual of $E(z,D)$ with mode (0,1,2) (last row, red.) is very
    large.} It seems that the mode $2$ impacts the optimization of mode 0 and
  1. However, as these three mode are decoupled, this effects should not happen.
\item \RED{ We need to require that $A_{ij} \ge 0$}, otherwise the residual
  $E'(z,D,A)$ will be much larger, which is about $11.6$ for mode (0,1,2) with
  keyframe unchanged, and $2.6\times 10^8$ when keyframe changed (These data is
  not shown in the table).
\end{itemize}

\paragraph{Problems } There are three problems need to be discussed according to
these results,
\begin{itemize}
\item When keyframe changed, the difference between $E'(z,D,A)$,
  $E'(z,D,\Lambda)$ and $E(z,D)$ is small. Does this means that optimizing $K$
  makes limited help on reducing the control forces?
\item Why the residual of $E(z,D)$ for modes (0,1,2) is large?
\item Why we need to constraint $A_{ij} \ge 0$?
\end{itemize}

\section{Optimize $E(z,D,A)$  Iteratively}
In this experiment, we will optimize $E(z,D,A)$ iteratively, with
$K=A^TA$. That is, we first optimize $E(z)$, then fix $z$, and optimize for $D$
and $A$. We make this experiment iteratively, and hope it can convergent to some
materials with small residuals.

\section{More experiments}
\subsection{Optimize Basis $\hat{W}$}
If we can obtain the desired results in the above experiments, we can try to
introduce $\hat{W}$ as variables in the optimization process, and use the
"Material Estimation" method in our proposal to solve this optimization
problem. And in this experiment, we can try to use some keyframes that can not
be defined by the initial $\hat{W}$, and we hope that the optimized $\hat{W}$
can well define these shapes.

\subsection{Partial constraints}
We can try to use some partial constraints $Cz = u^c$ instead of the keyframes,
and then solve for the optimal material.


%% references
% \begin{thebibliography}{99}
% \bibitem{sig2011} Fast simulation of skeleton-driven deformable body
%   characters.
% \end{thebibliography}

\end{document}

% \paragraph{Results} 
% We found that the residual $E(z)$ is still large when we use $\Lambda =
% \Lambda'$. This because the input animation are produced by an implicit
% integration method, which will introduce numerical damping. While for the
% discretion of $E(z)$, we use  This is
% method will not introduce numerical damping as implicit integration, and they
% are different. The solution is to introduce a diagonal damping matrix $\RED{D}$
% as additional variables in the optimization.

% \begin{center}
%   \begin{tabular}{ | l | l |  l | l | l |}
%     \hline
% 	mode & $E(z)$ & $E(z,D)$ & $E(z,D,\Lambda)$ & $E(z,D,\Lambda)$ ($z= 0.8z'$)& \\
%     \hline
% 	&  \\ \hline
% 	&  \\ \hline
%   \end{tabular}
% \end{center}
