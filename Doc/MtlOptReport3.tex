\documentclass[9pt,twocolumn]{extarticle}

\usepackage[hmargin=0.5in,tmargin=0.5in]{geometry}
\usepackage{amsmath,amssymb}
\usepackage{times}
\usepackage{graphicx}
\usepackage{subfigure}

\usepackage{cleveref}
\usepackage{color}
\newcommand{\TODO}[1]{\textcolor{red}{#1}}

\newcommand{\FPP}[2]{\frac{\partial #1}{\partial #2}}
\newcommand{\argmin}{\operatornamewithlimits{arg\ min}}
\author{Siwang Li}

\title{Report for Space-time Material Optimization}

%% document begin here
\begin{document}
\maketitle

\setlength{\parskip}{0.5ex}
\begin{abstract}
  Given a motion equation, the corresponding initial elastic materials, and some
  partial constraints provided by the user, the approach proposed here can
  generate an animation sequence satisfied these partial constraints, with
  minimal control forces and consistent effects. The novelties of this work
  include, an efficient method for the space-time elastic materials optimization
  in the subspace and an efficient Reduced RS model reduction method to support
  large deformation and promise tightly positional constraints. What's more, by
  providing a novel method for basis optimization(selection/combination), this
  approach can also automatically find a compact subspace to represent the
  resulting animation sequence.
\end{abstract}

\section{Background}\label{sec:background}
Firstly, our framework is based on space-time optimization on a set of decoupled
linear motion equations, which is closely related to modal analysis. Secondly,
we adopt the RS-coordinates to avoid the artifacts under large deformation. In
this section, we briefly review these techniques.

\subsection{Modal analysis}
The linear motion equation of an object is
\begin{equation} \label{linear_eq}
  \tilde{M}\ddot{u} + (\alpha_mI+\alpha_k\tilde{K})\dot{u} + \tilde{K}u = f_{ext}(t)
\end{equation}
where $u\in R^{3n}$ represents the displacements of all the $n$ nodes of the
object, $\tilde{M},\tilde{K} \in R^{3n\times 3n}$ are the mass and stiffness
matrix respectively, $\alpha_m,\alpha_k$ are the damping coefficients, and
$f_{ext}(t)$ is the external force at time $t$.

By using modal analysis method, motion equation (\ref{linear_eq}) can be reduced
and decoupled. To achieve this, we solve a general eigenvalue problem
\begin{equation} \label{general_eig}
  \tilde{K}\phi = \lambda \tilde{M}\phi
\end{equation}
If we preserve the first $r_w$ eigenvectors and eigenvalues to obtain
$W=(\phi_1,\cdots,\phi_{r_w})$, and $\Lambda=diag(\lambda_1,\lambda_{r_w})$,
then replace $u=Wz$ in equation (\ref{linear_eq}) and pre-multiply it with
$W^T$, finally we obtain the decoupled reduced motion equations
\begin{equation} \label{ma_eq}
  \ddot{z} + (\alpha_mI+\alpha_k\Lambda)\dot{z} + \Lambda{z} = w(t)
\end{equation}
Here, $w(t)=W^Tf_{ext}(t)$ represents the forces in the subspace.

\subsection{RS method}
Given a displacement $z\in R^{r_w}$ in the frequency subspace, there are many
method to construct the displacements $u(z)\in R^{3n}$ in full space, such as
linear mapping (e.g $u = Wz$), modal warping, or RS coordinates. Experiments
show that, when the deformation is large, only the RS coordinates method can
produce plausible results. In the following, we briefly review the RS method,
and we refer it as full-RS here. Using full-RS, we compute $u(z)$ by solving
\begin{equation}\label{full_rs_min}
  \min_{u} E_{rs}(u)=\frac{1}{2}\sum_{e\in T_{all}}\|(Gu)_e + I - \exp({y_e^w})(I+y_e^s)\|_2^2|V_e|
\end{equation}
where $T_{all}$ denotes all of the tetrahedrons, $G$ is the discrete gradient
operator with respect to the rest shape, $|V_e|$ is the volume of tetrahedron
$e$, $y_e=(y_e^w,y_e^s)$ is the RS coordinates of tetrahedron $e$. We denote the
RS-coordinates $y=(y_1,\cdots,y_{|T_{all}|})$, and compute it using
$y=\hat{W}z=(QW)z$, where $Q$ is a sparse matrix which maps the basis $W$ to the
RS-basis $\hat{W}$. Then the solution of (\ref{full_rs_min}) can be found by
solving
\begin{equation} \label{rs}
  Au = (VG)^Tb(y(z)).
\end{equation}
where $A=G^TV^2G$ is a constant sparse matrix,
$V=diag(\sqrt{V_1},\sqrt{V_1},\cdots,\sqrt{V}_{|T_{all}|})$ with each entry
repeated $9$x. Each $9\times1$ block of the long vector $b(y(z))\in
R^{9|T_{all}|}$ with respect to tetrahedron $e$ is expressed as a row-major
$3\times 3$ matrix,
\begin{equation}
  b_e(y_e(z)) = \sqrt{V_e}(\exp({y_e^w})(I+y_e^s)-I)
\end{equation}

\section{Reduced RS method}\label{sec:reduced-rs-method}
As in our spacetime optimization with positional constraints(introduced in
section \ref{sec:mater-optim}), we need to compute $u_j(z)$ and
$\frac{\partial{u_j(z)}}{\partial{z}}$ for each constrained node $j$
frequently. Using the full RS method introduced above, these computation would
be expensive. In this section, we presents a reduced RS method to improve the
performance.

Firstly, we assume $u$ can be well approximated by using some basis matrix $B\in
R^{3n\times r_b}$, then we replace $u = Bq$ into (\ref{rs}), and multiply
$B^T$in both sides to obtain
\begin{equation} \label{reduced_rs1} B^T(ABq) = B^T(VG)^Tb(y(z))
\end{equation} Let $P = (B^TAB)^{-1}B^T(VG)^T$, then we can compute the reduced
coordinates $q$ using
\begin{equation} \label{reduced_rs2} 
  q = Pb(y) = \sum_{e\in{T_{all}}}P_e{b_e(y_e(z))}
\end{equation} where $P_e\in R^{r_b\times 9}$ is some constant matrix according
to tetrahedron $e$.

Then, we approximate $q$ using the cubature scheme, and obtain
\begin{equation} \label{reduced_rs3} 
  q \approx \sum_{e\in{T_{cub.}}}w_eP_e{b_e(y_e(z))}
\end{equation} 
where $w_e$ is the cubature weights, and $T_{cub.}$ is the set of the selected
cubature tetrahedrons. Finally, we can compute the displacements $u$ using
\begin{equation} \label{reduced_rs4}
  u(z) = B \sum_{e\in{T_{cub.}}}w_eP_e{b_e(y_e(z))}
\end{equation} 
This approach is much faster than solving the large linear system (\ref{rs}).

\subsection{Performance analysis} 
To compute $u_j(z)$ and $\frac{\partial{u_j(z)}}{\partial{z}}$ for some node $j$
with the Reduced-RS method, we only need to compute $y_e(z)$ and $b_e(y_e)$ for
the sampled tetrahedrons (usually less than 100), and then make some small
matrix-vector production to obtain the results. While using full-RS method, to
compute $u_j(z)$, one needs to compute $y$ and $b$ for all the tetrahedrons, and
finally solve a large sparse linear system to compute $u$ for all the nodes, and
select $u_j(z)$ from them. And use full-RS method to compute
$\frac{\partial{u_j(z)}}{\partial{z}}$ would be much expensive than compute
$u_j(z)$. Thus, our reduced method should be much efficient than full-RS method,
especially when $u_j(z)$ and $\frac{\partial{u_j(z)}}{\partial{z}}$ need to be
computed frequently, such as the sparse positional constraints in our spacetime
optimization introduced bellow.

\section{Material optimization}\label{sec:mater-optim}
In the following, we will introduce our spacetime material optimization scheme.
Firstly, we choose the basis $B,W$ and cubature samples $T_{cub}$ to be large
enough to represent the desired deformations. We can use modal derivative method
to obtain $B$, and use modal analysis method to produce $W$. Then we further
assume that, the resulting animation can usually be represented by some much
smaller basis $W'=WS$, where $S\in R^{r_w\times r_s}$ is a dense matrix, and
$r_s$ is much smaller than $r_w$. For example, we can choose $r_w=80$, while
$r_s=30$. There is no necessary to require that $W'$ to be mass-orthogonal, as
the mass matrix is unknown.

Now, in the subspace with basis in $W'$, we assume that the motion of the object
to be edited can be described by this decoupled motion equation
\begin{equation} \label{ma_eq_1}
  \ddot{z} + (\alpha_mI+\alpha_k\Lambda)\dot{z} + \Lambda{z} = w(t)
\end{equation}
with $w(t)$ is the control forces.

Given some constraints, we define our spacetime material optimization problem as
\begin{eqnarray}\label{mtl-opt}
  &&\argmin_{z,\Lambda,S} E_w(z,\Lambda)+\gamma E_c(S,z)+\mu E_s(S)\\
  &&\mbox {subject to } \lambda_i \ge 0
\end{eqnarray}

Here, $\gamma$ and $\mu$ are the penalties, and $E_w$ is the residual for the
control forces $w(t)$, $E_c$ represents the residual of the constraints, while
$E_s$ is a regularization term for $S$. We will define each of these energies in
the following. 

\paragraph{Residual of control forces.} We define $E_w$ using discrete spacetime
constraints, and we further scale the traditional spacetime-constraints
formulation by time step as mentioned in the appendix
\ref{sec:scale-time-step}. Finally we obtain
\begin{equation} \label{ew}
  E_w(z,\Lambda) = \frac{1}{2}\sum_{i=1}^{T-2}\|\hat{z}_i+(\alpha_mI+\alpha_k\Lambda)(z_{i+1}-z_{i-1})+\Lambda z_i\|_2^2
\end{equation}
where $\hat{z}_i=z_{i+1}-2z_{i}+z_{i-1}$, and $T$ is the total number of frames.

\paragraph{Residual for constraints. } The residual for the partial constraints
can be defined as
\begin{equation} \label{ec}
  E_c(z,S) = \frac{1}{2}\sum_{(i,j)\in \mathbb{C}}\|u_{i,j}(z_i,S)-u^c_{i,j}\|_2^2
\end{equation}
where $\mathbb{C}$ contains all the constrained nodes $j$ at frame $i$, while
$u^c_{i,j}$ denotes the target displacements. We can compute $u_{i,j}(z_i,S)$
using our Reduced-RS method efficiently. 

What's more, we also found that it is necessary to scale each column of the
basis $\hat{W}$ using ${(\lambda_i^{0})^{-\frac{1}{2}}}$, where $\lambda_i^{0}$
is obtained by solving the general eigenvalue problem (\ref{general_eig}) with
the initial materials.  Finally we obtain
\begin{equation}
  u_{i,j}(S,z_i)=B_j\sum_{e\in T_{cub}}w_eP_eb_e(\hat{W}_e(\Lambda^{(0)})^{-\frac{1}{2}}Sz_i)
\end{equation}
where $T_{cub}$ denotes the sampled tetrahedrons of the cubature, $w_e$ is the
cubature weighting, $B_j\in R^{3\times r_b}$ is the basis for node $j$, $P_e\in
R^{r_b\times9}$ is a constant matrix for tetrahedron $e$, and $\hat{W}_e\in
R^{12\times r_w}$ is the sub-matrix of the RS-basis $\hat{W}$.

Sometimes, the artist hope to design or modify the animation using keyframes.
Usually it is inefficient to constrained all the nodes. In such case, one can
choose to constraint only the nodes of the selected tetrahedrons in the
cubature process.

\paragraph{Regularization of $S$. }
We penalize $S$ using
\begin{equation} \label{reg_s}
  E_s(S) = \sum_{i=0}^{r_w-1}\sum_{j=0}^{r_s-1} S^2_{i j}
\end{equation}.

\section{Numerical method}\label{sec:numer-optim}
Firstly, we fix $\Lambda=\Lambda^{(0)}, S=S^{(0)}$, and
solve (\ref{mtl-opt}) to obtain $z^{(0)}$, then in $k$-th outer iteration, we
solve (\ref{mtl-opt}) for $\Lambda^{(k+1)},S^{(k+1)},z^{(k+1)}$ as following,
\begin{itemize}
\item Fix $z=z^{(k)},\Lambda=\Lambda^{k}$, optimize for $S^{(k+1)}$.
\item Fix $z=z^{(k)},S=S^{(k+1)}$, optimize for $\Lambda^{(k+1)}$.
\item Fix $\Lambda=\Lambda^{(k+1)},S=S^{(k+1)}$, optimize for $z^{(k+1)}$.
\end{itemize}

We solve a general eigenvalue problem using the initial full stiffness matrix
and mass matrix to obtain the initial value $\Lambda^{(0)}$, and set
$S^{(0)}=(I_{r_s\times r_s},O)^T$. In the following, we will present the details
of the three steps in each outer iteration.

\subsection{Optimization of $z$}
We use Newton method to optimize for $z$, and we approximate the hessian using
\begin{equation} 
  H_z = H_{w,z}+\gamma J_{c,z}^TJ_{c,z}
\end{equation}
where
\begin{eqnarray}
  H_{w,z} &=& \frac{\partial^2{E_w}}{\partial^2{z}}\\
  J_{c,z} &=& \frac{\partial{\tilde{u}_c}}{\partial{z}}
\end{eqnarray}
Here, $\tilde{u}_c(z,S)$ is a long vector contains all $u_{i,j}(S,z_i)$. The computation
of $H_z$ is very efficient, and the computational cost of $J_{c,z}$ depends on
the number of the constrained nodes. When keyframes are used, there maybe
hundreds of constrained nodes, and the computational cost for $J_{c,z}$ would be
much more expensive than computing $H_{w,z}$.

\subsection{Optimization of $\Lambda$}
When $z$ is fixed, $E_c(\Lambda)$ is a quadratic function of $\Lambda$, and its
minimization can be obtained by solving a small linear equation. If the
resulting $\lambda_i<0$, we simply reset $\lambda_i=0$. 

\subsection{Optimization of $S$}
As both the hessian $\frac{\partial^2{E_c}}{\partial^2{S}}$ and its
approximation $J_{c,s}^TJ_{c,s}$ (where $J_{c,s} =
\frac{\partial{\tilde{u}_c}}{\partial{S}}$) are large dense matrices, we should
consider the hessian-free approaches. In our implementation, we adopt the LBFGS
approach.

\appendix
\section{Scale of time step}\label{sec:scale-time-step}
In our experiments, we found that the results are significantly effected by the
time step $h$, and the resulting animation is hard to control. Thus we scale
$\alpha_m,\alpha_k$ and $\Lambda$ by $h$ as following,
\begin{eqnarray}
 \alpha'_m &=& \frac{1}{h}\alpha_m  \\
 \alpha'_k &=& h\alpha_k  \\
 \Lambda' &=& \frac{1}{h^2}\Lambda
\end{eqnarray}
On the other hand, we also hope that $h$ will not effect the function value of
$E_w$, thus we scale $E_w$ by $h^4$, finally, we obtain
\begin{eqnarray}
  E_w(z,\Lambda)&=&\frac{h^4}{2}\sum_{i=1}^{T-2}\|\frac{1}{h^2}\hat{z}_{i}+\frac{1}{h}(\alpha'_mI+\alpha'_k\Lambda')(z_{i+1}-z_{i-1})+\Lambda'
  z_i\|_2^2 \nonumber\\
  &=&\frac{1}{2}\sum_{i=1}^{T-2}\|\hat{z}_{i}+(\alpha_mI+\alpha_k\Lambda)(z_{i+1}-z_{i-1})+\Lambda
  z_i\|_2^2
\end{eqnarray}
In this formulation, we totally remove $h$ from the energy function. 

\end{document}
