\documentclass[9pt,twocolumn]{extarticle}

\usepackage[hmargin=0.5in,tmargin=0.5in]{geometry}
\usepackage{amsmath,amssymb}
\usepackage{times}

\usepackage{cleveref}
\usepackage{color}
\newcommand{\TODO}[1]{\textcolor{red}{#1}}
\newcommand{\RED}[1]{\textcolor{red}{#1}}

\newcommand{\FPP}[2]{\frac{\partial #1}{\partial #2}}
\newcommand{\argmin}{\operatornamewithlimits{arg\ min}}
\author{Siwang Li}

\title{Implementation of Material Optimization using adjoint method}

%% document begin here
\begin{document}
\maketitle

\setlength{\parskip}{0.5ex}

\section{Introduction}
The optimization problem is
\begin{equation} \label{obj}
  \argmin_{w,\alpha_m,\alpha_k,\Lambda} \frac{1}{2}\int_{0}^T \| w(t) \|_2^2dt +
  \gamma_c \| Cu(z) - u^c \|_2^2 + \gamma_k\| z_i - z_i^k \|_2^2
\end{equation}
subject to
\begin{eqnarray}
  \ddot{z} + (\alpha_m+\alpha_k \Lambda)\dot{z} + \Lambda z &=& w \label{eq_con}
\end{eqnarray}
Here, we apply the partial constraints using penalty method, with
$\gamma_c,\gamma_k$ are the penalty parameters.

First, we fix $\alpha_m,\alpha_k,\Lambda$ and optimize for $w$ using nonlinear
optimization method, such as nonlinear-CG or BFGS method, and we use adjoint
method to compute the gradient. Then, we fix $z$ and optimize for
$\alpha_m,\alpha_k,\Lambda$. The iterations continues until the changes of $z$
is small than some tolerance.

\section{Optimize control forces}
Here, we fixed the materials, and optimize for $w$.  We use adjoint method to
compute the gradient. Here, as the motion equation are decoupled, here we
consider only one mode. Let $q = [v,z]^T$, and the integration function for step
$k+1$ is
\begin{equation} \label{mq2}
  q_{k+1} = f_k(q_k,w_k) = Gq_{k} + Sw_{k}
\end{equation}
and $q_0 = f_0 = (v_0,z_0)$, where
\begin{equation} \label{ab}
  G = \left( \begin{array}{cc}
      \gamma&-h\lambda \gamma \\
      h \gamma& 1-h^2\lambda \gamma
    \end{array} \right) \mbox{, }
  S = \left( \begin{array}{c}
      h\gamma\\
      h^2\gamma
    \end{array} \right)
\end{equation}
and $\gamma = \frac{1}{((h \alpha_k + h^2)\lambda + h \alpha_m + 1)}$.  Let
$\bold{q} = [q_0,\cdots,q_{T-1}]$, and $\bold{f} = [f_0,\cdots,f_{T-1}]$, then
we have $\bold{q} = \bold{f}(\bold{q},\bold{w})$. By using adjoint method, we
obtain
\begin{equation} \label{grad}
  \frac{dE}{dw} = (\frac{\partial{\bold{f}}}{\partial{w}})^T\bold{r} +
  \frac{\partial{E}}{\partial{w}} =
  \left( \begin{array}{c}
      S^Tr_0 + w_0\\
      \vdots\\
      S^Tr_{T-1} + w_{T-1}
    \end{array} \right)
\end{equation}
where we compute the adjoint vector $\bold{r}$ using
\begin{equation} \label{ad}
  r_{i} = (\frac{\partial{f_i}}{\partial{q_i}})^Tr_{i+1} +
  \frac{\partial{E}}{\partial{q_i}} = 
  G^Tr_{i+1} + \left( \begin{array}{c}
      0\\
      \RED {\frac{\partial{E}}{\partial{z_i}}}
    \end{array} \right)
\end{equation}

\subsection{Compute ${\frac{\partial{E}}{\partial{z_i}}}$} 
\subsubsection{Key frames} 
\begin{equation} \label{key_grad}
  {\frac{\partial{E}}{\partial{z_i}}} = \gamma_k(z_i-z_k)
\end{equation}

\subsubsection{Partial constraints}
The full RS method is equal to solve a linear system
\begin{equation} \label{rs}
  Au = b(y(z)).
\end{equation}
where $y(z) = \tilde{y} + \hat{W}z$. Let $B$ be some linear basis matrix, and
$P=(B^TAB)^{-1}B^T$, then by using the Cubature scheme, obtain
\begin{equation} \label{reduced_rs3}
  u \approx B\sum_{e\in{S}}w_eP_e{b_e(y(z))}
\end{equation}
Then we can compute the displacement $u_i(z)$ of node $i$ using
\begin{equation}
  u_i(z) = (B_iP)b(\tilde{y} + \hat{W}z)
\end{equation}
and the gradient is
\begin{equation}
  \frac{\partial{u_i}}{\partial{z}} = P_i \frac{\partial{b}}{\partial{y}}
  \hat{W} = P_i
  \left( \begin{array}{c}
      \frac{\partial{b}}{\partial{y_0}} \hat{W}_0\\
      \vdots\\
      \frac{\partial{b}}{\partial{y_N}} \hat{W}_N
    \end{array} \right)_{9N\times r_2} = 
  \sum(P_{ij}\frac{\partial{b}}{\partial{y_j}})\hat{W}_j
\end{equation}
where $P_{ij}\in R^{3\times 9}$, and
\begin{equation}
  b_j(y_j) = \sqrt{V_j}(exp(y_j^w)(I+y_j^e)-I)
\end{equation}
and $\frac{\partial{b_j}}{\partial{y_j}}$ is a $9\times9$ matrix. Finally we
compute 
\begin{equation} \label{compPePz}
  {\frac{\partial{E}}{\partial{z_i}}} =
  (\frac{\partial{u}}{\partial{z_i}})^Tu' =
  \sum\hat{W}_j^T(\frac{\partial{b}}{\partial{y_j}})^TP_{ij}^Tu'
\end{equation}
where $u' = C^TCu-C^Tu^c$.

\section{Optimize material}
The implicit integration method is
\begin{equation} \label{imp_integ}
  \left\{ \begin{array}{rl}
      v_{k+1} &= v_k + h(w_k-(\alpha_m+\alpha_kK)v_{k+1}-Kz_{k+1})\\
      z_{k+1} &= z_k + h v_{k+1}
    \end{array} \right.
\end{equation}
Then we have
\begin{equation} \label{opt_w_eq}
  hw_k = v_{k+1}-v_k + h (\alpha_m+\alpha_kK)v_{k+1} + hKz_{k+1}
\end{equation}
We define $K=A^TA$, then the optimization problem is
\begin{equation} \label{opt_w}
  \argmin_{\alpha_k,\alpha_m,A} \frac{1}{2}\sum_{k=0}^{T-2}\|v_{k+1}-v_k + h (\alpha_m+\alpha_kA^TA)v_{k+1} + hA^TAz_{k+1}\|_2^2
\end{equation}
Let
\begin{eqnarray}\label{VZ}
 V_0 &=& (v_0,\cdots,v_{T-2}) \nonumber \\
 V_1 &=& (v_1,\cdots,v_{T-1}) \nonumber \\
 Z_1 &=& (z_1,\cdots,z_{T-1}) \nonumber 
\end{eqnarray}
Then the objective function is 
\begin{equation} \label{obj_w}
  \Phi = \frac{1}{2}\| (\alpha_m+\alpha_kA^TA)(hV_1)+(A^TA)(hZ_1)+V_1-V_0\|_2^2
\end{equation}
\subsection{Gradient}
\begin{eqnarray*}
  E &=& hV_1 \\
  F &=& (\alpha_kA^TA)(hV_1) + (A^TA)(hZ_1)+V_1-V_0\\
  \Phi &=& \frac{1}{2}\| \alpha_mE + F \|_2^2 \\
  &=&\frac{1}{2}(\alpha_m^{i{ }k}E_{k{ }j}+F_{i{ }j})^2\\
  \frac{\partial{\Phi}}{\partial{\alpha_m^{l{ }l}}} &=&
  E_{k{ }j}(\alpha_m^{i{ }k}E_{k{ }j}+F_{i{ }j})\delta_l^i\delta_l^k\\
  &=&E_{l{ }j}F_{l{ }j} + \alpha_m^{l{ }l}E_{l{ }j}E_{l{ }j}\\
  &=&\mbox{row}(E,l)\cdot \mbox{row}(F,l) + \alpha_m^{l{ }l}
  \mbox{row}(E,l)\cdot \mbox{row}(E,l)\\
  G &=& (A^TA)(hV_1)\\
  H &=& (\alpha_mhV_1)+(A^TA)(hZ_1)+V_1-V_0\\
  \Phi &=& \frac{1}{2}\|\alpha_kG+H\|_2^2\\
  \frac{\partial{\Phi}}{\partial{\alpha_k^{l{ }l}}} &=&
  \mbox{row}(G,l)\cdot \mbox{row}(H,l) + \alpha_k^{l{ }l}\mbox{row}(G,l)\cdot
  \mbox{row}(G,l)\\
\end{eqnarray*}

\begin{eqnarray*}
  J &=& \alpha_k\\
  L &=& hZ_1\\
  M &=& \alpha_m(hV_1)+(V_1-V_0)\\
  \Phi &=& \frac{1}{2}\| J(A^TA)E + (A^TA)L + M \|_2^2\\
  \Phi &=& \frac{1}{2}(A_{si}A_{st}L_{th}+A_{lj}A_{lk}J_{ik}E_{jh}+M_{ih})^2\\
  \frac{\partial{\Phi}}{\partial{A_{xy}}}
  &=&A_{st}L_{th}(A_{si}A_{st}L_{th}+A_{lj}A_{lk}J_{ik}E_{jh}+M_{ih})\delta_x^s\delta_y^i+\\
  &&A_{si}L_{th}(A_{si}A_{st}L_{th}+A_{lj}A_{lk}J_{ik}E_{jh}+M_{ih})\delta_x^s\delta_y^t+\\
  &&A_{lk}J_{ik}E_{jh}(A_{si}A_{st}L_{th}+A_{lj}A_{lk}J_{ik}E_{jh}+M_{ih})\delta_x^l\delta_y^j+\\
  &&A_{lj}J_{ik}E_{jh}(A_{si}A_{st}L_{th}+A_{lj}A_{lk}J_{ik}E_{jh}+M_{ih})\delta_x^l\delta_y^k\\
  &=&A_{xt}L_{th}(A_{xy}A_{xt}L_{th}+A_{lj}A_{lk}J_{yk}E_{jh}+M_{yh})+\\
  &&A_{xi}L_{yh}(A_{xi}A_{xy}L_{yh}+A_{lj}A_{lk}J_{ik}E_{jh}+M_{ih})+\\
  &&A_{xk}J_{ik}E_{yh}(A_{si}A_{st}L_{th}+A_{xy}A_{xk}J_{ik}E_{yh}+M_{ih})+\\
  &&A_{xj}J_{iy}E_{jh}(A_{si}A_{st}L_{th}+A_{xj}A_{xy}J_{iy}E_{jh}+M_{ih})\\
  &=&  (AL)_{xh}((A^TAL)_{yh}+(E^TA^TAJ^T)_{hy}+M_{yh})+\\
  &&A_{xi}L_{yh}((A^TAL)_{ih}+(E^TA^TAJ^T)_{hi}+M_{ih})+\\
  &&(AJ^T)_{xi}E_{yh}((A^TAL)_{ih}+(E^TA^TAJ^T)_{hi}+M_{ih})+\\
  &&(AE)_{xh}J_{iy}((A^TAL)_{ih}+(E^TA^TAJ^T)_{hi}+M_{ih})\\
  &=& ((A^TALL^TA^T)_{yx}+(ALE^TA^TAJ^T)_{xy}+(ALM^T)_{xy})+\\
  &&((AA^TALL^T)_{xy}+(LE^TA^TAJ^TA^T)_{yx}+(AML^T)_{xy})+\\
  &&((AJ^TA^TALE^T)_{xy}+(EE^TA^TAJ^TJA^T)_{yx}+(AJ^TME^T)_{xy})+\\
  &&((J^TA^TALE^TA^T)_{yx}+(AEE^TA^TAJ^TJ)_{xy}+(AEM^TJ)_{xy})
\end{eqnarray*}

\subsection{Gradient of scalar $\alpha$}
\begin{eqnarray*}
  E &=& hV_1 \\
  F &=& (\alpha_kA^TA)(hV_1) + (A^TA)(hZ_1)+V_1-V_0\\
  \Phi &=& \frac{1}{2}\| \alpha_mE + F \|_2^2 \\
  &=&\frac{1}{2}(\alpha_mE_{i{ }j}+F_{i{ }j})^2\\
  \frac{\partial{\Phi}}{\partial{\alpha_m}} &=&
  E_{i{ }j}(\alpha_mE_{i{ }j}+F_{i{ }j})\\
  &=&E:(\alpha_mE+F)\\
  G &=& (A^TA)(hV_1)\\
  H &=& (\alpha_mhV_1)+(A^TA)(hZ_1)+V_1-V_0\\
  \Phi &=& \frac{1}{2}\|\alpha_kG+H\|_2^2\\
  \frac{\partial{\Phi}}{\partial{\alpha_k}} &=&G:(\alpha_kG+H)
\end{eqnarray*}


% And the gradient of this function is
% \begin{eqnarray}\label{par_alpha}
%   \frac{\partial{\Phi}}{\partial{\alpha_m}} &=& \\
%   \frac{\partial{\Phi}}{\partial{\alpha_k}} &=& \\
%   \frac{\partial{\Phi}}{\partial{A}} &=& 
% \end{eqnarray}
% And finally we use LBFGS or CG method to solve it.

%% references
% \begin{thebibliography}{99}
% \bibitem{sig2011} Fast simulation of skeleton-driven deformable body
%   characters.
% \end{thebibliography}

\end{document}