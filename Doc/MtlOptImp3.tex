\documentclass[twocolumn,a4paper]{article}
\usepackage{graphicx}
\usepackage{amsfonts}
\usepackage{color}
\usepackage{lineno}
\setlength{\columnsep}{10pt}
\setlength{\oddsidemargin}{0pt}
\setlength{\topmargin}{0pt}
\setlength{\headheight}{0pt}
\setlength{\headsep}{0pt}
\setlength{\marginparsep}{0pt}
\setlength{\marginparwidth}{0pt}
\addtolength{\voffset}{-50pt}
\setlength{\textheight}{26.8cm}
\author{Siwang Li}

\title{Elastic Material Optimization}

%% document begin here
\begin{document}
\maketitle

\section{Summary}


\section{Input} 
In the following experiment, we use an implicit integrator instead of an
analysis function to produce the input animations.

\section{Experiments}
In the first experiment, we check the numerical damping of the implicit
integration method. We use one mode to produce the input animation using an
implicit integrator, then optimize with and without damping $d$ respectively. We
record the residuals, and draw the input sequence $z_i$, $z(k,d)$ for each case.

In the second experiment, we use $r>1$ modes to produce the input animation, and
optimize for $D(d)=diag(d_1,\cdots,d_r)$, and $K(k)$(dense, symmetric). Then we
diagonalize $K$ to obtain $\hat{K}$ and $\hat{D}$, and compute the residuals.

In the third experiment, we check the whole approximation residuals with an
animation sequence in full space as input. We produce this sequence using a RS
simulator, without constraints or external forces.

The fourth experiment is similar to the above one, except that we produce the
input animation using a full StVK simulator.

After making the above experiments, we can begin to check the impact of the
external forces and constraints. And firstly, we check impact of the external
forces, and we also start from a single mode.

\end{document}