\documentclass[9pt,twocolumn]{extarticle}

\usepackage[hmargin=0.5in,tmargin=0.5in]{geometry}
\usepackage{amsmath,amssymb}
\usepackage{times}
\usepackage{graphicx}
\usepackage{subfigure}

\usepackage{cleveref}
\usepackage{color}
\newcommand{\TODO}[1]{\textcolor{red}{#1}}

\newcommand{\FPP}[2]{\frac{\partial #1}{\partial #2}}
\newcommand{\argmin}{\operatornamewithlimits{arg\ min}}
\author{Siwang Li}

\title{cubature scheme}

%% document begin here
\begin{document}
\maketitle

\setlength{\parskip}{0.5ex}

\section{Cubature}
The optimized cubature, proposed by An et al\cite{cubature2008}, constructs an
approximation of the reduced forces based on a best subset selection
problem. And we found that this approach be used to find the sample and
weightings of our reduced RS method. Let $f_e(z) = P_eb_e(y_e(z))$, then we have
\begin{equation} \label{reduced_rs6}
  q(z) = \sum_{e\in\mathcal{T}}f_e(y_e(z)),
\end{equation}
and our Reduce RS method is
\begin{equation} \label{reduced_rs7}
  q(z) = \sum_{e\in\mathcal{T}_{\text{cub}}}\!\!\omega_e f_e(y_e(z)).
\end{equation}
If we view $q(z)$ as the reduced internal forces item in \cite{cubature2008},
and $z$ as the reduced coordinates with respect to $q$, then we can use the
similar cubature scheme to obtain $\omega_e$ by solving a nonnegative least
sequence problem, and find the cubature samples $\mathcal{T}_{\text{cub}}$ by
using a greedy estimation scheme.

Given a set of reduced coordinates $\{z_i\}_{i=1}^{N_{tr}}$ (here, $N_{tr}$ is
the total number of training set, and we use a modal-analysis simulator
\cite{ma} to generate them), we construct $u_i(z_i)$ using full rs method, then
compute $q_i=B^Tu_i$. Then we find the cubature samples
$\mathcal{T}_{\text{cub}}$ by using a greedy estimation scheme iteratively, and
at each step, we estimate the nonnegative weights $\omega_e$ by solving a
nonnegative least squares problem $\bold{A}\bold{w}=\bold{b}$, where
$\omega=(\omega_1,\hdots,\omega_{|\mathcal{T}_{\text{cub}}|})^T$, and
\begin{equation} \label{nls}
  \bold{A} = \left( \begin{array}{ccccc}
      \frac{f_1^{(1)}}{\|q^{(1)}\|}&\hdots&\frac{f_{|\mathcal{T}_{\text{cub}}|}^{(1)}}{\|q^{(1)}\|}\\
      \vdots &&\vdots\\
      \frac{f_{1}^{(N_{tr})}}{\|q^{(N_{tr})}\|}&\hdots&\frac{f_{|\mathcal{T}_{\text{cub}}|}^{(N_{tr})}}{\|q^{(N_{tr})}\|}
    \end{array} \right)
\end{equation}

\begin{equation} \label{nls}
  \bold{b} = \left( \begin{array}{ccccc}
      \frac{q^{(1)}}{\|q^{(1)}\|}\\
      \vdots\\
      \frac{q^{(N_{tr})}}{\|q^{(N_{tr})}\|}
    \end{array} \right)
\end{equation}

For our examples, we can usually choose $N_{tr}=600$, and
$\mathcal{T}_{\text{cub}}$, and the cubature process can usually finished in a
few minutes. In \cite{cubature2013}, the authors propose a more fast cubature
scheme, and we can consider to use their approach to accelerate this
pre-computation.

\section{Other problems}
\begin{itemize}
\item line 255, the iterations for solving $S$ using LBFGS depends on the
  constraints, and currently, no warm start is used, simply set $S^{(0)}=(I,O)$,
  and we use the ALGLIB Free Edition(GPL) with single-threaded\cite{lbfgs}.
\item line 259, the left-hand matrix for solving $\Lambda$ is diagonal, thus we
  solve it directly without using any specific linear solver.
\item line 285, the computational cost for $B$ depends on the complexity of the
  tetrahedron mesh and the basis we need in $B$. For large models with $r_B=80$,
  it may takes about 20 minutes. A more efficient approach has been proposed by
  \cite{cubature2013}, by using this approach, it is possible to obtain $B$ in a
  few minutes.
\end{itemize}


%% references
\begin{thebibliography}{99}
\bibitem{cubature2008} Optimizing Cubature for Efficient Integration of Subspace
  Deformations, siggraph asia, 2008.
\bibitem{cubature2013} An Efficient Construction of Reduced Deformable Objects,
  siggraph asia, 2013.
\bibitem{ma} Interactive Deformation Using Modal Analysis with Constraints,
  Graphics Interface 2003.
\bibitem{lbfgs} http://www.alglib.net/.
\end{thebibliography}

\end{document}