\documentclass[9pt,twocolumn]{extarticle}

\usepackage[hmargin=0.5in,tmargin=0.5in]{geometry}
\usepackage{amsmath,amssymb}
\usepackage{times}
\usepackage{graphicx}
\usepackage{subfigure}

\usepackage{cleveref}
\usepackage{color}
\newcommand{\TODO}[1]{\textcolor{red}{#1}}

\newcommand{\FPP}[2]{\frac{\partial #1}{\partial #2}}
\newcommand{\argmin}{\operatornamewithlimits{arg\ min}}
\author{Siwang Li}

\title{introduction for the results}

%% document begin here
\begin{document}
\maketitle

\setlength{\parskip}{0.5ex}

\section{Recover the material of simulated results}
We use two examples to demonstrate the ability our approach for recovering
non-uniform material.  Firstly, we use a full stvk simulator in full space
to produce a animation sequence for each model. For the simulation, the material
of both model are non-uniform distributed: one part of the beam is much softer
than the rest part, and the tail of the dinosaur is much softer than the rest
part. Then, we use uniform material as the initial value, and sampled some
points(according to the cubature results) in the first 100 frames of the
simulated results, and use them as positional constraints in our material
optimization method. Finally we use the optimize material to produce the rest
100 animation frames, and compare them with the input animation. We also produce
the results with no material optimization, (e.g optimize for z only). It can be
seen that our method can recover the elastic material.

\section{Track prescribed path}
In this example, we demonstrate that our approach can track the prescribed
path, and what's more, keep details of the input animation. The input animation
is a flower, which is simulated using a full stvk simulator using some random. 

%% references
% \begin{thebibliography}{99}
% \bibitem{sig2011} Fast simulation of skeleton-driven deformable body
%   characters.
% \end{thebibliography}

\end{document}